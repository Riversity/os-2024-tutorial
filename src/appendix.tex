\chapter{细节说明}
\section{环境配置}
\subsubsection{EDK2}\label{appendix:launchuefi}
以下是一个编译EDK2的OVMF和相关efi的示例脚本:
\begin{lstlisting}[language=bash]
#!/bin/bash

# 获取脚本所在目录的绝对路径
SCRIPT_DIR="$(cd "$(dirname "${BASH_SOURCE[0]}")" && pwd)"
# 获取项目根目录(假设脚本在 tools/scripts 目录下)
PROJECT_ROOT="$(cd "${SCRIPT_DIR}/../.." && pwd)"

cd ${PROJECT_ROOT}/edk2

export WORKSPACE=${PROJECT_ROOT}/edk2
export EDK_TOOLS_PATH=${PROJECT_ROOT}/edk2/BaseTools
export CONF_PATH=${PROJECT_ROOT}/tools/config/edk2 # 默认是 edk2/Conf ,为了优雅这里把它单独提出来

source edksetup.sh

make -C BaseTools -j$(nproc)

build

build -p ShellPkg/ShellPkg.dsc

build -p YourPkg/YourPkg.dsc

\end{lstlisting}
以下是一个一键启动EDK2 UEFI环境的示例脚本:
\begin{lstlisting}[language=bash]
#!/bin/bash

# 获取脚本所在目录的绝对路径
SCRIPT_DIR="$(cd "$(dirname "${BASH_SOURCE[0]}")" && pwd)"
# 获取项目根目录(假设脚本在 tools/scripts 目录下)
PROJECT_ROOT="$(cd "${SCRIPT_DIR}/../.." && pwd)"

# 定义资源文件路径
OVMF_CODE="${PROJECT_ROOT}/edk2/Build/Ovmf3264/DEBUG_GCC5/FV/OVMF_CODE.fd"
OVMF_VARS="${PROJECT_ROOT}/edk2/Build/Ovmf3264/DEBUG_GCC5/FV/OVMF_VARS.fd"
# SHELL_EFI="${PROJECT_ROOT}/edk2/Build/Shell/DEBUG_GCC5/X64/ShellPkg/Application/Shell/EA4BB293-2D7F-4456-A681-1F22F42CD0BC/DEBUG/Shell.efi"
RAW_ACPIVIEW_EFI="${PROJECT_ROOT}/edk2/Build/Shell/DEBUG_GCC5/X64/ShellPkg/Application/AcpiViewApp/AcpiViewApp/DEBUG/AcpiViewApp.efi"
HELLO_WORLD_EFI="${PROJECT_ROOT}/edk2/Build/YourPkg/DEBUG_GCC5/X64/YourPkg/Application/HelloWorld/HelloWorld/DEBUG/HelloWorld.efi"
HALLO_WORD_EFI="${PROJECT_ROOT}/edk2/Build/YourPkg/DEBUG_GCC5/X64/YourPkg/Application/HalloWord/HalloWord/DEBUG/HalloWord.efi"
MY_ACPIVIEW_EFI="${PROJECT_ROOT}/edk2/Build/YourPkg/DEBUG_GCC5/X64/YourPkg/Application/AcpiView/AcpiView/DEBUG/AcpiView.efi"

# 检查上述文件是否存在
RESOURCE_LIST=("$OVMF_CODE" "$OVMF_VARS" "$RAW_ACPIVIEW_EFI" "$HELLO_WORLD_EFI" "$HALLO_WORD_EFI" "$MY_ACPIVIEW_EFI")

for RESOURCE in "${RESOURCE_LIST[@]}"; do
	echo "检查文件: $RESOURCE"
	if [ ! -f "$RESOURCE" ]; then
		echo "错误:$RESOURCE 不存在,请确认编译路径是否正确"
		exit 1
	fi
done
# 创建运行目录(如果不存在)
PLAYGROUND_DIR="${PROJECT_ROOT}/playground"
rm -rf "$PLAYGROUND_DIR"
mkdir -p "$PLAYGROUND_DIR"

# 复制OVMF变量文件(避免修改原始文件)
cp "$OVMF_VARS" "${PLAYGROUND_DIR}/OVMF_VARS.fd"

mkdir -p "$PLAYGROUND_DIR/uefi"
# cp "$SHELL_EFI" "${PLAYGROUND_DIR}/uefi/Origin_Shell.efi"
cp "$RAW_ACPIVIEW_EFI" "${PLAYGROUND_DIR}/uefi/O_AcpiViewApp.efi"
cp "$MY_ACPIVIEW_EFI" "${PLAYGROUND_DIR}/uefi/My_AcpiView.efi"
cp "$HELLO_WORLD_EFI" "${PLAYGROUND_DIR}/uefi/HelloWorld.efi"
cp "$HALLO_WORD_EFI" "${PLAYGROUND_DIR}/uefi/HalloWord.efi"

# 暂停
# read -p "按任意键继续..."

# 启动QEMU进入UEFI shell
qemu-system-x86_64 \
	-machine q35,accel=kvm \
	-m 8G \
	-smp 4 \
	-drive if=pflash,format=raw,unit=0,file="${OVMF_CODE}",readonly=on \
	-drive if=pflash,format=raw,unit=1,file="${PLAYGROUND_DIR}/OVMF_VARS.fd" \
	-drive file=fat:rw:"${PLAYGROUND_DIR}/uefi",format=raw,if=ide,index=0 \
	-nographic \
	-no-reboot \
	-serial mon:stdio

\end{lstlisting}