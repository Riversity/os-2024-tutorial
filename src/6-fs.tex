\chapter{文件系统}
文件是UNIX设计哲学中的一个重要组成部分,
一切皆文件的抽象使得用户调整系统的参数也变得更加简单。
因此,如何实现一个既能管理系统又能管理日常文件的文件系统便显得更加重要。
Linux通过一套统一的VFS(虚拟文件系统)抽象达成这一目标。
在本实践部分,你需要对VFS进行修改并且尝试使用文件系统。

\section{inode和扩展属性管理}

目标:读取设置inode的基本信息,引入新的\textbf{文件扩展属性(Extended file attributes)}。你需要实现一个\texttt{.c}的文件\textbf{直接}调用系统调用调整文件扩展属性。


测试:测试程序会先调用你的程序进行文件扩展属性的读写,随后和标准的文件扩展属性进行比对。你不需要在这个作业中修改内核。



\section{用户态文件系统}
由于文件系统在Linux中被编译为内核对象(kernel object),文件系统崩溃会传导到内核造成内核不可用。此外,内核对象的灵活性较差。
Linux给出了FUSE的服务,允许在用户态实现一个文件系统。

目标:使用FUSE实现GPT服务。声明一个GPTfs,其中的目录为每一个对话session,对话session文件夹下的input为本轮用户输入的prompt,output文件为GPT输出的结果。每一轮对话都是单轮对话,无需考虑记录上下文。具体而言,你需要实现:

\begin{itemize}
\item 实现一个名为\texttt{GPTfs}的FUSE文件系统。
\item 文件系统的根目录下,每次创建一个新目录即表示开启一个新的对话会话(session)。
\item 在每个对话session目录下,包含以下文件:
\begin{enumerate}
\item \texttt{input}:用于存放本轮用户输入的prompt。
\item \texttt{output}:用于存放GPT生成的回复结果。
\item \texttt{error}:用于存放网络连接错误等其他错误内容。
\end{enumerate}
\item 对\texttt{input}文件的写入操作应触发GPT处理,生成回复内容并写入对应的\texttt{output}文件。
\item 每一轮对话都是单轮对话,无需记录上下文。
\end{itemize}


测试:测试程序会写入input文件,随后你应当调用GPT服务输出,在这里测试程序会等待5秒后检查error和output。


\section{用户空间下的内存磁盘}
随着硬件的发展,内存的价格已经显著下降。
内存的访问速度非常快,可以作为一部分反复读写的文件的临时存储。
在这一个任务中,你需要从用户态划出一块区域用于磁盘存储。

目标:使用FUSE实现RAMfs功能,要求读写碎片尽可能少并且支持硬链接。

\subsection{接口要求}

\begin{itemize}
\item 实现FUSE文件系统需要的回调函数,包括但不限于:
\begin{enumerate}
\item \texttt{getattr}:获取文件或目录的属性。
\item \texttt{readdir}:读取目录内容。
\item \texttt{mknod}:创建文件。
\item \texttt{mkdir}:创建目录。
\item \texttt{unlink}:删除文件。
\item \texttt{rmdir}:删除目录。
\item \texttt{rename}:重命名文件或目录。
\item \texttt{link}:创建硬链接。
\item \texttt{open}、\texttt{read}、\texttt{write}:文件的打开、读取和写入操作。
\end{enumerate}
\item 设计高效的内存数据结构,管理文件系统的元数据和文件内容。
\item 实现对硬链接的支持,正确更新inode的引用计数,确保文件在所有链接被删除后才释放内存。
\item 处理并发的读写请求,确保数据的一致性和线程安全。
\end{itemize}

\subsection{测试}

\begin{itemize}
\item 基本的文件和目录操作测试:创建、读取、写入、删除、重命名。
\item 硬链接的创建和删除(inode引用计数的正确性以及文件内容的共享)。
\item 多线程高并发的读写操作。
\item 检查内存使用情况,评估内存泄露和文件碎片。
\end{itemize}

\subsection{提示}

\begin{itemize}
\item 可以参考内核中RAMfs的设计思路,但不得直接复制代码。
\item 注意内存的分配和释放,确保每次分配的内存都有对应的释放操作。
\item 使用锁或者其他同步机制来处理并发访问。
\item 充分测试各种边界条件,如大文件、小文件、深层目录结构等。
\end{itemize}
