\chapter{系统启动}
计算机上电后,系统会进行一系列初始化操作完成硬件设备的逐个启动。
早年这一个过程由执行每个设备上的ROM中存储的代码(一般称为固件firmware)结合BIOS完成。
随后,更加通用的UEFI被提出,UEFI的全称是Unified Extensible Firmware Interface,它是一套统一使用C语言编写在OS启动前的固件代码段。

\begin{figure}[h]
    \centering
    \includegraphics[width=\linewidth]{figure/mixed-figures/boot.pdf}
    \caption{UEFI启动过程}
    \label{fig:enter-label}
\end{figure}

这些固件代码段还会收集硬件信息,并且移交给操作系统。
早年的黑苹果、伪造OEM激活Windows 7都采用劫持UEFI启动后ACPI Table内容实现欺骗OS的效果。
在这个专题的实践中,我们需要体验读取、修改系统启动过程中的各个表,并且尝试传递更多的信息给操作系统。

\section{Read ACPI Table}
目标:修改EDK2的UEFI组件,在其中增加函数:\texttt{PrintAllACPITables}。该函数不传入任何参数,要求打印每一张表的地址、长度以及每一张其中所有的信息与校验信息。

可以参考EDK2下的\texttt{AcpiView.c} 中的 \texttt{AcpiView} 和 \texttt{EfiLocateFirstAcpiTable} 函数,但是不得直接复制。

此外,关于 ACPI 的更多信息,可以参考\href{https://uefi.org/sites/default/files/resources/ACPI_Spec_6.5a_Final.pdf}{这本手册}来学习,为了完成对应的任务,也许你只需要关注 ACPI 表的构成和 checksum 的含义即可。

具体测评指标:
\begin{itemize}
\item 输出所有ACPI表的完整地址映射;
\item 每个表的输出必须包含:物理地址(64位十六进制)、表长度(32位十进制)、OEM ID(6字节ASCII)、校验和(8位十六进制)。
\end{itemize}

为了能够在 UEFI Shell 中运行该函数,你可以学习如何自定义一个模块,或者在一个模块中添加应用。然后,你可以用这样的代码来测试你的程序:

\begin{lstlisting}[language=bash]
    cp edk2/Build/Shell/DEBUG_GCC5/X64/...{Your application efi file here} ./uefi/AcpiView.efi
    qemu-system-x86_64 \
      -drive if=pflash,format=raw,readonly=on,file={Your OVMF_CODE.fd here} \
      -drive if=pflash,format=raw,file={Your OVMF_VARS.fd here} \
      -drive file=fat:rw:./uefi,format=raw,if=ide,index=0 \
      -nographic \
      -no-reboot
\end{lstlisting}

为了自动运行你的程序,你可以在 \texttt{uefi} 文件夹中创建 \texttt{startup.nsh} 文件。该文件中的指令会在进入 UEFI Shell 时自动执行,参考内容如下:

\begin{lstlisting}[language=bash]
fs0:              # change to the filesystem where the acpi-viewer.efi is located
AcpiView.efi      # run the AcpiView.efi
reset             # shut down the VM (because -no-reboot is set)
\end{lstlisting}

\subsection{ACPI Table 结构}

ACPI Table 是一个树状结构,其第一张表是 RSDP(Root System Description Pointer),它里面存放了 RDST/XDST 的地址。RDST是32位地址,XDST是64位地址,其功能是一样的。
而 RDST/XDST 中记录了其他表的地址。

注意在 FADT(Fixed ACPI Description Table) 中有还记录 FACS 和 DSDT 的地址,这个两个表无法直接在 RDST/XDST 中找到。

\subsection{.efi 编译与 EDK2 辅助文件撰写}
为了编译出能在 UEFI 中运行的 .efi 格式组件,我们需要在编译时指定特殊的编译目标。

例如,在 Rust 中,你只需要在 rust-toolchain.toml 中指定
\texttt{targets = ["x86\_64-unknown-uefi"]}.

或者,使用 gcc 或 clang ,也可以通过一系列复杂的命令行参数达成这一目的。

当然,为了利用 EDK2 中已有的结构体、宏、函数定义等,使用 EDK2 环境进行组件编写是比较好的选择。
而且,也只需要 \texttt{source edksetup.sh; build -p MyACPIPkg/MyACPIPkg.dsc} 即可像其他Pkg一般完成编译。

唯一的障碍是,在 EDK2 中,除了 .c 代码以外,还需要撰写比较麻烦的 .dsc 与 .inf 文件。
其他组件(如\texttt{helloworld.c})的这两个文件可以作为撰写的参考,但由于环境复杂,文件内容相对繁多。但实际上,只有少数 field 需要定义和填充。

下面是一个Minimum Working Example: 

In MyACPI.inf :

\begin{lstlisting}
[Defines]
  INF_VERSION             = 0x00011451        # 此处出于一些规定需要≥0x00010005
  BASE_NAME               = MyACPI
  MODULE_TYPE             = UEFI_APPLICATION  # 否则无法生成.efi
  VERSION_STRING          = 1.0
  ENTRY_POINT             = MyACPIFunc        # 函数入口

[Sources]
  MyACPI.c                                    # 源代码相对.inf的路径

[Packages]
  MdePkg/MdePkg.dec
  MdeModulePkg/MdeModulePkg.dec

[LibraryClasses]
  UefiApplicationEntryPoint
  UefiLib
  PcdLib
\end{lstlisting}

In MyACPIPkg.dsc :

\begin{lstlisting}
[Defines]
  PLATFORM_NAME           = MyACPI
  PLATFORM_GUID           = XXXXXXXX-XXXXXXXX # 应由各位自行生成
  PLATFORM_VERSION        = 0.01
  OUTPUT_DIRECTORY        = Build/MyACPI      # .efi 输出目录
  SUPPORTED_ARCHITECTURES = IA32|X64|EBC|ARM|AARCH64|RISCV64|LOONGARCH64
  BUILD_TARGETS           = DEBUG|RELEASE

!include MdePkg/MdeLibs.dsc.inc
  
[LibraryClasses]
  UefiApplicationEntryPoint|MdePkg/Library/UefiApplicationEntryPoint/UefiApplicationEntryPoint.inf
  # ...
  # 此处取决于你 #include 的头文件等依赖库
  # 看 build 时报的缺文件错误信息即可定向修补
  # 比较取巧的方式是直接复制其他 .dsc 中的一坨
  
[Components]
  MyACPIPkg/MyACPI.inf                        # .inf 相对 edk2/ 根目录的路径
\end{lstlisting}

\section{Hack ACPI Table}
目标:修改EDK2的UEFI组件,在其中增加函数:\texttt{ChangeACPITable}。该函数传入一个\texttt{UINTN}的对象表明修改第几张表、输入\texttt{EFI\_ACPI\_DESCRIPTION\_HEADER*}表明待修改的数值,要求修改完成后打印每一张表的地址、长度以及每一张其中所有的信息与校验信息。

测试:通过引导Linux后读取对应的ACPI表检查输出进行验证。

具体评测指标:
\begin{itemize}
\item 修改\texttt{EFI\_ACPI\_DESCRIPTION\_HEADER}中所有字段(包括Signature保留字段)且自动生成正确校验和的比例达到100\%;
\item 系统重启3次后修改仍然有效;
\item 内核日志(\texttt{dmesg | grep ACPI})无CHECKSUM\_ERROR警告或其他ACPI警告。
\end{itemize}

\section{UEFI运行时服务}
目标:增加一个新的UEFI运行时服务,使得Linux系统可以调用这个新的服务导出硬件运行时信息。

提示:需要修改UEFI固件和Linux内核(sysfs部分),方便直接读取导出的结果。

测试:展示
