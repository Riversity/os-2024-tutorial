\documentclass[lang=cn,10pt]{elegantbook}

\title{计算机系统(2)实践手册}
% \subtitle{}

\author{ACM班课程助教}
\institute{ACM班,上海交通大学}
\date{2024-2025春季学期}
\version{1.1.1}
% \bioinfo{}{}

\definecolor{codebg}{rgb}{0.95,0.95,0.95}
\lstset{
    backgroundcolor=\color{codebg},
    basicstyle=\ttfamily\footnotesize,
    breaklines=true,
    frame=single,
    captionpos=b
}
\extrainfo{}

\setcounter{tocdepth}{3}

% \logo{logo-blue.png}
\cover{cover.jpg}

% 本文档命令
\usepackage{array}
\newcommand{\ccr}[1]{\makecell{{\color{#1}\rule{1cm}{1cm}}}}

% 修改标题页的橙色带
% \definecolor{customcolor}{RGB}{32,178,170}
% \colorlet{coverlinecolor}{customcolor}

\begin{document}

\maketitle
\frontmatter

\tableofcontents
\input{src/0-start}
\mainmatter
\input{src/1-intro}
\chapter{项目环境准备}

\section{基准代码}
本系列的基准代码规则如下:
\begin{enumerate}
    \item \textbf{Linux}: 任何体系结构下高于5.10.20版本的Linux内核都是可行的选项,你可以从\url{https://www.kernel.org/}中选择合适的版本。对于\texttt{initramfs},你可以选择任何一个发行版。如果你选择直接使用qcow或者其他完全预制好的Ubuntu镜像,你需要手动在内部更换内核以确保你的代码能够生效。
    \item \textbf{QEMU}: 版本>=7。
    \item \textbf{EDK2}: 跟随\url{https://github.com/tianocore/edk2}选择合适的版本。其中Ubuntu\_GCC5并不意味着GCC版本必须是5。
\end{enumerate}

\section{在QEMU中启动Linux镜像}
\begin{remark}
简单起见你可以直接使用预制作好的\texttt{qcow} Ubuntu镜像作为磁盘镜像由qemu直接引导,具体教程:\url{https://documentation.ubuntu.com/public-images/en/latest/public-images-how-to/launch-qcow-with-qemu/}。如果你选用这个方案,你只需要确保QEMU正常安装即可。
\end{remark}

\begin{remark}
    本区域中.x.x部分请填入自己对应的版本号。
\end{remark}



\subsection{QEMU安装(二选一)}
\subsubsection{软件包安装}
\begin{lstlisting}[language=bash]
sudo apt-get install qemu-system-x86
qemu-system-x86_64 --version
\end{lstlisting}

\subsubsection{源码编译安装}
\begin{lstlisting}[language=bash]
# 安装依赖
sudo apt-get install git libglib2.0-dev libfdt-dev libpixman-1-dev zlib1g-dev ninja-build

# 下载编译
mkdir ~/kvm && cd ~/kvm
wget https://download.qemu.org/qemu-7.2.0.tar.xz
tar -xf qemu-7.2.0.tar.xz
cd qemu-7.2.0
./configure --target-list=x86_64-softmmu
make -j$(nproc)
\end{lstlisting}

\subsubsection{常见依赖问题}
\begin{itemize}
    \item \textbf{ERROR: Cannot find Ninja} → 执行 \texttt{sudo apt install ninja-build}
    \item \textbf{ERROR: glib-2.56 required} → 执行 \texttt{sudo apt install libglib2.0-dev}
    \item \textbf{Dependency "pixman-1" not found} → 执行 \texttt{sudo apt install libpixman-1-dev}
\end{itemize}

\subsection{Linux内核编译}
\begin{lstlisting}[language=bash]
# 安装依赖
sudo apt-get install git fakeroot build-essential ncurses-dev xz-utils libssl-dev bc flex libelf-dev bison 
mkdir ~/kernel && cd ~/kernel
wget https://git.kernel.org/pub/scm/linux/kernel/git/stable/linux.git/snapshot/linux-5.x.x.tar.gz
tar -xf linux-5.x.x.tar.gz
cd linux-5.x.x

make defconfig    # 使用默认配置
make -j$(nproc)   # 开始编译
\end{lstlisting}

\textbf{编译输出文件}: \texttt{arch/x86/boot/bzImage}

在编译过程中,可能会发生编译错误的情况。这可能是 GCC 版本过高导致的。这里给出一种可行的组合: Linux-5.19.17, 使用 GCC-12 进行编译,操作如下:

\begin{lstlisting}[language=bash]
sudo apt install gcc-12 g++-12 # 安装gcc-12
sudo update-alternatives --install /usr/bin/gcc gcc /usr/bin/gcc-12 70
sudo update-alternatives --install /usr/bin/g++ g++ /usr/bin/g++-12 70 # 切换 gcc 默认版本
\end{lstlisting}
经过这些操作后,应该可以正常编译 Linux 内核了。

\subsection{BusyBox}
\subsubsection{直接安装}
直接使用你的开发环境自带的包管理器安装busybox,然后在后面制作initramfs的过程中添加这两步:

先把busybox拷贝进initramfs:
\begin{lstlisting}[language=bash]
cp $(which busybox) "${INITRAMFS_DIR}/bin/busybox"
\end{lstlisting}

然后在系统启动脚本里添加:
\begin{lstlisting}[language=bash]
/bin/busybox --install -s /bin
\end{lstlisting}
\subsubsection{手动编译安装}
这里可以采用其他发行版(例如Ubuntu发行版等)。
\begin{lstlisting}[language=bash]
mkdir ~/kvm && cd ~/kvm
wget https://busybox.net/downloads/busybox-1.x.x.tar.bz2
tar -xf busybox-1.x.x.tar.bz2
cd busybox-1.x.x

# 配置命令
make menuconfig
# 选中: Settings > Build Options > [*] Build static binary
# 取消: Shells > [ ] Job control

make -j$(nproc) && make install
\end{lstlisting}
如果在编译过程中出现错误( TCA\_CBQ\_MAX undeclared ),一种解决方式是直接删除出错的 “tc.c” 文件。之后就可以正常编译了(版本 1.32.0 可以用此方法解决)。

\subsection{EDK2编译(简述)}
EDK2的可编译源码在git submodule里,因此相对简单可靠的方式就是直接按照官网所说的那样:
\begin{lstlisting}[language=bash]
git clone -b stable/202408 https://github.com/tianocore/edk2.git
cd edk2
git submodule update --init
\end{lstlisting}

在编译之前,需要在系统里安装一些基础的包(例如base-devel uuid、nasm、acpica),然后就可以编译BaseTools。

\begin{lstlisting}[language=bash]
    sudo apt install iasl nasm
\end{lstlisting}

在笔者实践的过程中,遇到了默认安装的 nasm 版本过低,导致后续编译出错的问题。读者可参考后文 \ref{compile-edk2} 的内容,自行编译 nasm。

\begin{lstlisting}[language=bash]
source edksetup.sh
make -C BaseTools -j$(nproc)
\end{lstlisting}
然后编辑\texttt{Conf/target.txt}(参考\url{https://github.com/tianocore/tianocore.github.io/wiki/How-to-build-OVMF}):
\begin{lstlisting}
ACTIVE_PLATFORM       = OvmfPkg/OvmfPkgIa32X64.dsc
TARGET_ARCH           = IA32 X64
TOOL_CHAIN_TAG        = GCC5
\end{lstlisting}
最后,执行:
\begin{lstlisting}[language=bash]
build
\end{lstlisting}
然后就可以在\texttt{Build/Ovmf3264/DEBUG\_GCC5/FV/}目录找到\texttt{OVMF\_CODE.fd}和\texttt{OVMF\_VARS.fd}两个编译出的OVMF文件。(这里的Ovmf3264和DEBUG\_GCC5和编译选项有关,也是在\texttt{Conf/target.txt}里面配置。)

\subsection{制作initramfs}
\subsubsection{创建启动脚本}
\begin{lstlisting}[language=bash, language=bash, showstringspaces=false]
cd busybox-1.x.x/_install/
mkdir -p proc sys dev tmp
echo -e '#!/bin/sh\nmount -t proc none /proc\nmount -t sysfs none /sys\nmount -t tmpfs none /tmp\nmount -t devtmpfs none /dev\necho "Hello Linux!"\nexec /bin/sh' > init
chmod +x init
\end{lstlisting}

\subsubsection{打包文件系统}
\begin{lstlisting}[language=bash]
find . -print0 | cpio --null -ov --format=newc | gzip -9 > ../initramfs.cpio.gz
\end{lstlisting}

\subsection{启动QEMU}
\subsubsection{图形界面启动}
\begin{lstlisting}[language=bash]
qemu-system-x86_64 \
    -kernel ./kernel/linux-5.x.x/arch/x86/boot/bzImage \
    -initrd ./kvm/busybox-1.x.x/initramfs.cpio.gz \
    -append "init=/init"
\end{lstlisting}

\subsubsection{字符界面启动}
\begin{lstlisting}[language=bash, showstringspaces=false]
qemu-system-x86_64 \
    -kernel ./kernel/linux-5.x.x/arch/x86/boot/bzImage \
    -initrd ./kvm/busybox-1.x.x/initramfs.cpio.gz \
    -nographic \
    -append "init=/init console=ttyS0"
\end{lstlisting}
如果使用EDK2的话,可以在参数里加上两行:
\begin{lstlisting}[language=bash]
    -drive if=pflash,format=raw,unit=0,file="${OVMF_CODE}",readonly=on \
    -drive if=pflash,format=raw,unit=1,file="${PLAYGROUND_DIR}/OVMF_VARS.fd" \
\end{lstlisting}

\subsection{验证}
\begin{itemize}
    \item 检查内核版本: \texttt{uname -a}
    \item 查看启动参数: \texttt{cat /proc/cmdline}
    \item 测试基本命令: \texttt{ls}, \texttt{mount}
\end{itemize}

\section{Linux环境部署 EDK2 开发环境}

python的版本要求>=3.11,

\subsubsection{下载源码}
\begin{lstlisting}[language=bash]
git clone https://github.com/tianocore/edk2.git
cd edk2
git submodule update --init  #大概需要几分钟
cd ..
\end{lstlisting}

\subsubsection{网上找到的可能需要安装的一堆东西} \label{compile-edk2}

\begin{lstlisting}[language=bash]
sudo apt-get install build-essential uuid-dev
sudo apt-get install uuid-dev nasm bison flex
sudo apt-get install libx11-dev x11proto-xext-dev libxext-dev
\end{lstlisting}

将 NASM 更新到 2.15.x 以上版本,当前的edk2项目需要依赖nasm2.15以上版本,否则编译会报错
nasm各版本下载链接:http://www.nasm.us/pub/nasm/releasebuilds/?C=M;O=D
(特别注意,推荐下载 .tar.xz 格式的 source code, .zip 解压后的文件似乎都是 CRLF 格式,可能会导致编译错误)

\begin{lstlisting}[language=bash]
cd nasm-2.16
./autogen.sh
./configure
make
make install
nasm --version
\end{lstlisting}

\subsubsection{make一下BaseTools}

然后cd到edk2/BaseTools下
\begin{lstlisting}[language=bash]
make
\end{lstlisting}

显示ok则成功。
注意这里所有的文件应该是LF,不能是CRLF,不然会出问题。可以先检查一下。

然后到edk2的conf目录下,创建target.txt(template在readme里面)
然后修改:
TARGET\_ARCH           = X64
TOOL\_CHAIN\_TAG        = GCC5

\subsubsection{编译UEFI模拟器和UEFI工程}
先到edk2目录下,然后执行
\begin{lstlisting}[language=bash]
sourse edksetup.sh
build
\end{lstlisting}

然后到target.txt里面改一下:
ACTIVE\_PLATFORM       = OvmfPkg/OvmfPkgX64.dsc

\subsubsection{运行}
\paragraph{最小化运行}
在一开始的任务中,完全没有必要拉起一个完整的系统环境,甚至连Linux系统内核镜像都不需要启动,只需要全程在UEFI环境里执行。

此时的UEFI环境,简单来讲由两部分构成:
\begin{enumerate}
	\item OVMF文件:用于在虚拟机中,提供UEFI固件,从而启动UEFI环境。
	\item 若干.efi文件:UEFI环境下的可执行文件
\end{enumerate}

之前所说的那个`build`指令,用处就是编译OVMF固件。你可能还希望使用这个命令编译一个更强大的UEFI Shell
\begin{lstlisting}[language=bash]
build -p ShellPkg/ShellPkg.dsc
\end{lstlisting}

然后你会找到一个\texttt{AcpiViewApp.efi}文件(可以使用\texttt{find . -name "AcpiViewApp.efi"}来寻找它),把它挂载进文件系统,你就可以在UEFI Shell里使用它来执行EDK2预先写好的一个AcpiView。详细的示例,参考\hyperref[appendix:launchuefi]{附录}

\paragraph{完整运行}
首先下载一个ubuntu的iso镜像,比如ubuntu-24.04.1-live-server-amd64.iso
然后找到自己的OVMF.fd,一般在edk2/Build/OvmfX64/DEBUG\_GCC5/FV/OVMF.fd下
下面的bios就是跑一下自己编译出的UEFI模拟器,cdrom就是iso镜像,m 4096是内存大小,enable-kvm是开启虚拟化加速
正常来说sudo可以不加,m可以不加,kvm的指令不需要加,但是我不加的话会报错,所以加上了
\begin{lstlisting}[language=bash]
sudo qemu-system-x86_64 -m 4096 -cdrom ~/ubuntu-24.04.1-live-server-amd64.iso -bios ~/OVMF.fd -enable-kvm
\end{lstlisting}

\
\chapter{系统启动}
计算机上电后,系统会进行一系列初始化操作完成硬件设备的逐个启动。
早年这一个过程由执行每个设备上的ROM中存储的代码(一般称为固件firmware)结合BIOS完成。
随后,更加通用的UEFI被提出,UEFI的全称是Unified Extensible Firmware Interface,它是一套统一使用C语言编写在OS启动前的固件代码段。

\begin{figure}[h]
    \centering
    \includegraphics[width=\linewidth]{figure/mixed-figures/boot.pdf}
    \caption{UEFI启动过程}
    \label{fig:enter-label}
\end{figure}

这些固件代码段还会收集硬件信息,并且移交给操作系统。
早年的黑苹果、伪造OEM激活Windows 7都采用劫持UEFI启动后ACPI Table内容实现欺骗OS的效果。
在这个专题的实践中,我们需要体验读取、修改系统启动过程中的各个表,并且尝试传递更多的信息给操作系统。

\section{Read ACPI Table}
目标:修改EDK2的UEFI组件,在其中增加函数:\texttt{PrintAllACPITables}。该函数不传入任何参数,要求打印每一张表的地址、长度以及每一张其中所有的信息与校验信息。

可以参考EDK2下的\texttt{AcpiView.c} 中的 \texttt{AcpiView} 和 \texttt{EfiLocateFirstAcpiTable} 函数,但是不得直接复制。

此外,关于 ACPI 的更多信息,可以参考\href{https://uefi.org/sites/default/files/resources/ACPI_Spec_6.5a_Final.pdf}{这本手册}来学习,为了完成对应的任务,也许你只需要关注 ACPI 表的构成和 checksum 的含义即可。

具体测评指标:
\begin{itemize}
\item 输出所有ACPI表的完整地址映射;
\item 每个表的输出必须包含:物理地址(64位十六进制)、表长度(32位十进制)、OEM ID(6字节ASCII)、校验和(8位十六进制)。
\end{itemize}

为了能够在 UEFI Shell 中运行该函数,你可以学习如何自定义一个模块,或者在一个模块中添加应用。然后,你可以用这样的代码来测试你的程序:

\begin{lstlisting}[language=bash]
    cp edk2/Build/Shell/DEBUG_GCC5/X64/...{Your application efi file here} ./uefi
    qemu-system-x86_64 \
      -drive if=pflash,format=raw,readonly=on,file={Your OVMF_CODE.fd here} \
      -drive if=pflash,format=raw,file={Your OVMF_VARS.fd here} \
      -drive file=fat:rw:./uefi,format=raw,if=ide,index=0\
      -nographic 
    \end{lstlisting}

\subsection{ACPI Table 结构}

ACPI Table 是一个树状结构,其第一张表是 RSDP(Root System Description Pointer),它里面存放了 RDST/XDST 的地址。RDST是32位地址,XDST是64位地址,其功能是一样的。
而 RDST/XDST 中记录了其他表的地址。

注意在 FADT(Fixed ACPI Description Table) 中有还记录 FACS 和 DSDT 的地址,这个两个表无法直接在 RDST/XDST 中找到。

\section{Hack ACPI Table}
目标:修改EDK2的UEFI组件,在其中增加函数:\texttt{ChangeACPITable}。该函数传入一个\texttt{UINTN}的对象表明修改第几张表、输入\texttt{EFI\_ACPI\_DESCRIPTION\_HEADER*}表明待修改的数值,要求修改完成后打印每一张表的地址、长度以及每一张其中所有的信息与校验信息。

测试:通过引导Linux后读取对应的ACPI表检查输出进行验证。

具体评测指标:
\begin{itemize}
\item 修改\texttt{EFI\_ACPI\_DESCRIPTION\_HEADER}中所有字段(包括Signature保留字段)且自动生成正确校验和的比例达到100\%;
\item 系统重启3次后修改仍然有效;
\item 内核日志(\texttt{dmesg | grep ACPI})无CHECKSUM\_ERROR警告或其他ACPI警告。
\end{itemize}

\section{UEFI运行时服务}
目标:增加一个新的UEFI运行时服务,使得Linux系统可以调用这个新的服务导出硬件运行时信息。

提示:需要修改UEFI固件和Linux内核(sysfs部分),方便直接读取导出的结果。

测试:展示

\input{src/4-privileged}
\input{src/5-mm}
\chapter{文件系统}
文件是UNIX设计哲学中的一个重要组成部分,
一切皆文件的抽象使得用户调整系统的参数也变得更加简单。
因此,如何实现一个既能管理系统又能管理日常文件的文件系统便显得更加重要。
Linux通过一套统一的VFS(虚拟文件系统)抽象达成这一目标。
在本实践部分,你需要对VFS进行修改并且尝试使用文件系统。

\section{inode和扩展属性管理}

目标:读取设置inode的基本信息,引入新的\textbf{文件扩展属性(Extended file attributes)}。你需要实现一个\texttt{.c}的文件\textbf{直接}调用系统调用调整文件扩展属性。


测试:测试程序会先调用你的程序进行文件扩展属性的读写,随后和标准的文件扩展属性进行比对。你不需要在这个作业中修改内核。



\section{用户态文件系统}
由于文件系统在Linux中被编译为内核对象(kernel object),文件系统崩溃会传导到内核造成内核不可用。此外,内核对象的灵活性较差。
Linux给出了FUSE的服务,允许在用户态实现一个文件系统。

目标:使用FUSE实现GPT服务。声明一个GPTfs,其中的目录为每一个对话session,对话session文件夹下的input为本轮用户输入的prompt,output文件为GPT输出的结果。每一轮对话都是单轮对话,无需考虑记录上下文。具体而言,你需要实现:

\begin{itemize}
\item 实现一个名为\texttt{GPTfs}的FUSE文件系统。
\item 文件系统的根目录下,每次创建一个新目录即表示开启一个新的对话会话(session)。
\item 在每个对话session目录下,包含以下文件:
\begin{enumerate}
\item \texttt{input}:用于存放本轮用户输入的prompt。
\item \texttt{output}:用于存放GPT生成的回复结果。
\item \texttt{error}:用于存放网络连接错误等其他错误内容。
\end{enumerate}
\item 对\texttt{input}文件的写入操作应触发GPT处理,生成回复内容并写入对应的\texttt{output}文件。
\item 每一轮对话都是单轮对话,无需记录上下文。
\end{itemize}


测试:测试程序会写入input文件,随后你应当调用GPT服务输出,在这里测试程序会等待5秒后检查error和output。


\section{用户空间下的内存磁盘}
随着硬件的发展,内存的价格已经显著下降。
内存的访问速度非常快,可以作为一部分反复读写的文件的临时存储。
在这一个任务中,你需要从用户态划出一块区域用于磁盘存储。

目标:使用FUSE实现RAMfs功能,要求读写碎片尽可能少并且支持硬链接。

\subsection{接口要求}

\begin{itemize}
\item 实现FUSE文件系统需要的回调函数,包括但不限于:
\begin{enumerate}
\item \texttt{getattr}:获取文件或目录的属性。
\item \texttt{readdir}:读取目录内容。
\item \texttt{mknod}:创建文件。
\item \texttt{mkdir}:创建目录。
\item \texttt{unlink}:删除文件。
\item \texttt{rmdir}:删除目录。
\item \texttt{rename}:重命名文件或目录。
\item \texttt{link}:创建硬链接。
\item \texttt{open}、\texttt{read}、\texttt{write}:文件的打开、读取和写入操作。
\end{enumerate}
\item 设计高效的内存数据结构,管理文件系统的元数据和文件内容。
\item 实现对硬链接的支持,正确更新inode的引用计数,确保文件在所有链接被删除后才释放内存。
\item 处理并发的读写请求,确保数据的一致性和线程安全。
\end{itemize}

\subsection{测试}

\begin{itemize}
\item 基本的文件和目录操作测试:创建、读取、写入、删除、重命名。
\item 硬链接的创建和删除(inode引用计数的正确性以及文件内容的共享)。
\item 多线程高并发的读写操作。
\item 检查内存使用情况,评估内存泄露和文件碎片。
\end{itemize}

\subsection{提示}

\begin{itemize}
\item 可以参考内核中RAMfs的设计思路,但不得直接复制代码。
\item 注意内存的分配和释放,确保每次分配的内存都有对应的释放操作。
\item 使用锁或者其他同步机制来处理并发访问。
\item 充分测试各种边界条件,如大文件、小文件、深层目录结构等。
\end{itemize}

\chapter{网络与外部设备}

\section{tcpdump和socket管理}
Linux内核提供了抓包(caputure packets)等服务作为内核网络套件的一部分。本节的目标包括对tcpdump进行定制化修改,以及对内核Socket的管理进行改进,从而实现更高的安全性和基于线程级的公平性(per-thread socket fairness)。

\subsection{功能目标}
\begin{enumerate}
    \item \textbf{定制化tcpdump抓包:}在tcpdump基础上进行修改,允许用户在不依赖tcpdump现有参数的情况下,自行传入过滤规则或者其他控制参数,以满足特定的抓包需求。
    \item \textbf{Socket公平性管理:}在内核中为每个线程提供独立的Socket资源上限或限流控制(fairness),确保多线程在执行网络操作时不会发生资源独占或过度使用的现象。
\end{enumerate}

\subsubsection{标准函数声明示例}
以下仅给出示例,用于说明如何编写对应接口的函数声明,具体细节可根据实际需求进行扩展。

\begin{lstlisting}[language=C, caption={示例:自定义tcpdump抓包接口}]
/**
 * @brief 使用自定义过滤规则对网络数据进行抓包
 * @param iface          需要进行抓包的网络接口名称
 * @param custom_filter  用户自定义的过滤表达式
 * @param buffer         存储抓取到的数据缓冲区
 * @param buffer_size    存储抓包数据的缓冲区大小
 * @return 成功时返回0,失败返回非0错误码
 */
int custom_tcpdump_capture(const char* iface, 
                           const char* custom_filter, 
                           void* buffer, 
                           size_t buffer_size);
\end{lstlisting}

\begin{lstlisting}[language=C, caption={示例:线程级别Socket公平性管理接口}]
/**
 * @brief 为特定线程配置Socket级别的公平管理策略
 * @param thread_id          需要配置的线程ID
 * @param max_socket_allowed 该线程允许打开的最大Socket数
 * @param priority_level     该线程的优先级(影响Socket分配策略)
 * @return 成功时返回0,失败返回非0错误码
 */
int configure_socket_fairness(pthread_t thread_id, 
                              int max_socket_allowed, 
                              int priority_level);
\end{lstlisting}

\subsection{测试}
\begin{itemize}
    \item \textbf{定制抓包正确性:}验证是否确实只抓取了满足自定义过滤规则的数据。  
    \item \textbf{多Socket公平性:}测试多线程同时创建并使用Socket时,是否能按照规划的最大数目和优先级进行分配,并观察线程间资源竞争的情况。
    \item \textbf{性能指标:}在系统负载较高时统计p99延迟(多Socket读写操作)。
\end{itemize}

\section{高速通信库:NCCL}
由于单节点计算能力的提升,分布式计算对通信带宽和延迟提出了极高要求。传统以太网通信常常无法满足此种高吞吐需求。NCCL (NVIDIA Collective Communications Library) 提供了GPU间高速通信接口,可广泛应用于深度学习训练场景。下文给出了相关参考文档以及源代码仓库:
\begin{itemize}
    \item \href{https://docs.nvidia.com/deeplearning/nccl/user-guide/docs/examples.html#communicator-creation-and-destruction-examples}{Nvidia NCCL examples}
    \item \href{https://github.com/NVIDIA/nccl}{Nvidia NCCL GitHub repo}
\end{itemize}

\subsection{功能目标}
\begin{enumerate}
    \item 在多GPU环境下,使用NCCL实现简单的信息传输(如广播、All-Reduce等),验证数据传输的正确性。
    \item 与以太网通信方式进行带宽和延迟对比,观察速度提升和正确性差异。
\end{enumerate}

\subsubsection{标准函数声明示例}
\begin{lstlisting}[language=C, caption={示例:NCCL通信例程接口}]
/**
 * @brief 使用NCCL执行多GPU数据广播操作
 * @param data         需要广播的数据缓冲区指针
 * @param count        数据元素的个数
 * @param root         广播发送方的GPU ID
 * @param comm         NCCL通信器
 * @return ncclSuccess表示成功,其他错误码请参照NCCL官方文档
 */
ncclResult_t nccl_broadcast_data(void* data, 
                                 size_t count, 
                                 int root, 
                                 ncclComm_t comm);
\end{lstlisting}

\subsection{测试项与测试方式}
\begin{itemize}
    \item \textbf{正确性测试:}不同GPU间进行广播、聚合(All-Reduce)后,结果是否一致,误差是否在可接受范围内(通常为浮点运算误差)。
    \item \textbf{性能测量:}分别采用NCCL和以太网(TCP/IP)进行跨节点通信,记录带宽、吞吐量、延迟等指标,对比二者差异。
\end{itemize}

\section{数据平面开发套件:DPDK}
DPDK( Data Plane Development Kit )是一套在用户态加速网络收发的开发工具包。它将网络协议栈从内核态卸载到用户态中运行,从而极大减少特权级切换(toggle between privileged levels)并提升网络吞吐量。本节在有限的复杂度下,演示如何利用DPDK实现多进程带宽平衡的QoS管理。

\subsection{功能目标}
\begin{enumerate}
    \item 在用户态基于DPDK实现最简化的网络收发功能。
    \item 设计一个QoS管理策略,使得多个进程共享网络带宽时,能根据设定的策略进行带宽分配并保证平衡。
\end{enumerate}

\subsubsection{标准函数声明示例}
\begin{lstlisting}[language=C, caption={示例:DPDK QoS管理接口}]
/**
 * @brief 在DPDK环境下为队列配置带宽限速
 * @param queue_id       DPDK端口队列编号
 * @param rate_limit     限速值(单位:Mbps)
 * @return 0表示成功,非0表示发生错误
 */
int dpdk_qos_configure_rate_limit(uint16_t queue_id, 
                                  uint32_t rate_limit);

/**
 * @brief 发送数据包入口函数
 * @param buffer         需要发送的包数据指针
 * @param len            包长度
 * @param queue_id       发送使用的DPDK端口队列编号
 * @return 发送成功的数据包数,出现错误返回负值
 */
int dpdk_send_packet(const uint8_t* buffer, 
                     size_t len, 
                     uint16_t queue_id);
\end{lstlisting}

\subsection{测试项与测试方式}
\subsubsection{测试项}
\begin{itemize}
    \item \textbf{基本发送接收功能:}在用户态能够正确收包、发包,确保功能可用。
    \item \textbf{多进程带宽平衡:}当多个进程请求同时发送数据时,观察各进程的实际带宽分配是否符合设定的QoS管理策略。
    \item \textbf{性能指标:}记录最大吞吐量、p99延迟,比较在无QoS管理与有QoS管理时的性能差异。
\end{itemize}

\subsubsection{测试方法}
\begin{enumerate}
    \item \textbf{功能正确性:}  设置单进程收发场景,发送固定大小的数据包,检查收发统计信息。
    \item \textbf{多进程带宽平衡:}  并行启动多个进程,各自不断发送数据包。使用dpdk\_qos\_configure\_rate\_limit函数为不同队列设置不同的速率限制(不同版本可能略有不同,可以使用自己期望的版本并且告知助教后进行调整),测试带宽使用情况是否符合预期配额。
    \item \textbf{压力测试:} 持续提高数据发送速率或缩短发包间隔,检测稳定性、丢包率和延迟情况,测试不同QoS设置的效果。
\end{enumerate}

\input{src/8-security_vm}
\input{src/25-mixed}

% \chapter{Elegant\LaTeX{} 系列模板介绍}

% Elegant\LaTeX{} 项目组致力于打造一系列美观、优雅、简便的模板方便用户使用。目前由 \href{https://github.com/ElegantLaTeX/ElegantNote}{ElegantNote},\href{https://github.com/ElegantLaTeX/ElegantBook}{ElegantBook},\href{https://github.com/ElegantLaTeX/ElegantPaper}{ElegantPaper} 组成,分别用于排版笔记,书籍和工作论文。强烈推荐使用最新正式版本!本文将介绍本模板的一些设置内容以及基本使用方法。如果您有其他问题,建议或者意见,欢迎在 GitHub 上给我们提交 \href{https://github.com/ElegantLaTeX/ElegantBook/issues}{issues} 或者邮件联系我们。

% 我们的联系方式如下,建议加入用户 QQ 群提问,这样能更快获得准确的反馈,加群时请备注 \LaTeX{} 或者 Elegant\LaTeX{} 相关内容。
% \begin{itemize}
%   \item 官网:\href{https://elegantlatex.org/}{https://elegantlatex.org/}(暂时歇业)
%   \item GitHub 地址:\href{https://github.com/ElegantLaTeX/}{https://github.com/ElegantLaTeX/}
%   \item Gitee 地址:\href{https://gitee.com/ElegantLaTeX}{https://gitee.com/ElegantLaTeX}
%   \item CTAN 地址:\href{https://ctan.org/pkg/elegantbook}{https://ctan.org/pkg/elegantbook}
%   \item 下载地址:\href{https://github.com/ElegantLaTeX/ElegantBook/releases}{正式发行版},\href{https://github.com/ElegantLaTeX/ElegantBook/archive/master.zip}{最新版}
%   \item 微博:Elegant\LaTeX{}(密码有点忘了)
%   \item 微信公众号:Elegant\LaTeX{}(不定期更新)
%   \item 用户 QQ 群:692108391(建议加群)
%   \item 邮件:\email{elegantlatex2e@gmail.com}
% \end{itemize}

% \section{模板安装与更新}

% 你可以通过免安装的方式使用本模板,包括在线使用和本地(文件夹内)使用两种方式,也可以通过 \TeX{} 发行版安装使用。

% \subsection{在线使用模板}

% 我们把三套模板全部上传到 \href{https://www.overleaf.com/}{Overleaf} 上了,网络便利的用户可以直接通过 Overleaf 在线使用我们的模板。使用 Overleaf 的好处是无需安装 \TeX{} Live,可以随时随地访问自己的文件。查找模板,请在 Overleaf 模板库里面搜索 \lstinline{elegantlatex} 即可,你也可以直接访问\href{https://www.overleaf.com/latex/templates?addsearch=elegantlatex}{搜索结果}。选择适当的模板之后,将其 \lstinline{Open as Template},即可把模板存到自己账户下,然后可以自由编辑以及与别人一起协作。更多关于 Overleaf 的介绍和使用,请参考 Overleaf 的\href{https://www.overleaf.com/learn}{官方文档}。

% \subsection{本地免安装使用}

% \textbf{免安装}使用方法如下:从 GitHub 或者 CTAN 下载最新版,严格意义上只需要类文件 \lstinline{elegantbook.cls}。然后将模板文件放在你的工作目录下即可使用。这样使用的好处是,无需安装,简便;缺点是,当模板更新之后,你需要手动替换 \lstinline{cls} 文件。

% \subsection{发行版安装与更新}

% 本模板测试环境为 
% \begin{enumerate}
%   \item Win10 + \TeX{} Live 2022;
%   \item Ubuntu 20.04 + \TeX{} Live 2022;
%   \item macOS Monterey + Mac\TeX{} 2022。
% \end{enumerate}

% \TeX Live/Mac\TeX{} 的安装请参考啸行的\href{https://github.com/OsbertWang/install-latex-guide-zh-cn/releases/}{一份简短的关于安装 \LaTeX{} 安装的介绍}。

% 安装 \TeX{} Live 之后,安装后建议升级全部宏包,升级方法:使用 cmd 或 terminal 运行 \lstinline{tlmgr update --all},如果 tlmgr 需要更新,请使用 cmd 运行 \lstinline{tlmgr update --self},如果更新过程中出现了中断,请改用 \lstinline{tlmgr update --self --all --reinstall-forcibly-removed} 更新,也即

% \begin{lstlisting}
% tlmgr update --self 
% tlmgr update --all
% tlmgr update --self --all --reinstall-forcibly-removed
% \end{lstlisting}

% 更多的内容请参考 \href{https://tex.stackexchange.com/questions/55437/how-do-i-update-my-tex-distribution}{How do I update my \TeX{} distribution?}

% \subsection{其他发行版本}

% 由于宏包版本问题,本模板不支持 C\TeX{} 套装,请务必安装 TeX Live/Mac\TeX{}。更多关于 \TeX{} Live 的安装使用以及 C\TeX{} 与 \TeX{} Live 的兼容、系统路径问题,请参考官方文档以及啸行的\href{https://github.com/OsbertWang/install-latex-guide-zh-cn/releases/}{一份简短的关于安装 \LaTeX{} 安装的介绍}。


% \section{关于提交}

% 出于某些因素的考虑,Elegant\LaTeX{} 项目自 2019 年 5 月 20 日开始,\textbf{不再接受任何非作者预约性质的提交}(pull request)!如果你想改进模板,你可以给我们提交 issues,或者可以在遵循协议(LPPL-1.3c)的情况下,克隆到自己仓库下进行修改。


% \chapter{ElegantBook 设置说明}

% 本模板基于基础的 book 文类,所以 book 的选项对于本模板也是有效的(纸张无效,因为模板有设备选项)。默认编码为 UTF-8,推荐使用 \TeX{} Live 编译。

% \section{语言模式}
% 本模板内含两套基础语言环境 \lstinline{lang=cn}、\lstinline{lang=en}。改变语言环境会改变图表标题的引导词(图,表),文章结构词(比如目录,参考文献等),以及定理环境中的引导词(比如定理,引理等)。不同语言模式的启用如下:
% \begin{lstlisting}
% \documentclass[cn]{elegantbook} 
% \documentclass[lang=cn]{elegantbook}
% \end{lstlisting}

% 除模板自带的两套语言设定之外,由\textbf{网友}提供的其他语言环境设置如下:
% \begin{itemize}
%   \item 由 \href{https://github.com/VincentMVV}{VincentMVV} 提供的意大利语翻译 \lstinline{lang=it},相关讨论见 \href{https://github.com/ElegantLaTeX/ElegantBook/issues/85}{Italian translation};
%   \item 由 \href{https://github.com/abfek66}{abfek66} 提供的法语翻译 \lstinline{lang=fr},相关讨论见 \href{https://github.com/ElegantLaTeX/ElegantBook/issues/85}{Italian translation};
%   % \item 由 \href{https://github.com/stultus}{stultus} 提供的马拉雅拉姆语翻译 \lstinline{lang=},相关讨论见 \href{https://github.com/ElegantLaTeX/ElegantBook/issues/90}{Malayalam translation};
%   \item 由 \href{https://github.com/inktvis75}{inktvis75} 提供的荷兰语翻译 \lstinline{lang=nl},相关讨论见 \href{https://github.com/ElegantLaTeX/ElegantBook/issues/108}{Dutch Translation};
%   \item 由 \href{https://github.com/palkotamas}{palkotamas} 提供的匈牙利语翻译 \lstinline{lang=hu},相关讨论见 \href{https://github.com/ElegantLaTeX/ElegantBook/issues/111}{Hungarian translation};
%   \item 由 Lisa 提供的德语翻译 \lstinline{lang=de},相关讨论见 \href{https://github.com/ElegantLaTeX/ElegantBook/issues/113}{Deutsch translation};
%   \item 由 Gustavo A. Corradi 提供的西班牙语的翻译 \lstinline{lang=es},相关讨论见 \href{https://github.com/ElegantLaTeX/ElegantBook/issues/133}{Spanish translation};
%   \item 由 \href{https://github.com/Altantsooj}{Altantsooj} 提供的蒙古语的翻译 \lstinline{lang=mn},相关讨论见 \href{https://github.com/ElegantLaTeX/ElegantBook/issues/137}{Mongolian translation};
%   \item 由 \href{https://github.com/inusturbo}{inusturbo} 提供的日本语的翻译 \lstinline{lang=jp},相关讨论见 \href{https://github.com/ElegantLaTeX/ElegantBook/issues/172}{Japanese Translation}。
% \end{itemize}



% \begin{remark}
% 以上各个语言的设定均为网友设定,我们未对上述翻译进行过校对,如果有问题,请在对应的 issue 下评论。并且,只有中文环境(\lstinline{lang=cn})才可以输入中文。
% \end{remark}

% \section{设备选项}
% 最早我们在 ElegantNote 模板中加入了设备选项(\lstinline{device}),后来,我们觉得这个设备选项的设置可以应用到 ElegantBook 中\footnote{不过因为 ElegantBook 模板封面图片的存在,在修改页面设计时,需要对图片进行裁剪。},而且 Book 一般内容比较多,如果在 iPad 上看无需切边,放大,那用户的阅读体验将会得到巨大提升。你可以使用下面的选项将版面设置为 iPad 设备模式\footnote{默认为 normal 模式,也即 A4 纸张大小。}
% \begin{lstlisting}
% \documentclass[pad]{elegantbook} %or
% \documentclass[device=pad]{elegantbook}
% \end{lstlisting}

% \section{颜色主题}

% 本模板内置 5 组颜色主题,分别为 \textcolor{structure1}{\lstinline{green}}\footnote{为原先默认主题。}、\textcolor{structure2}{\lstinline{cyan}}、\textcolor{structure3}{\lstinline{blue}}(默认)、\textcolor{structure4}{\lstinline{gray}}、\textcolor{structure5}{\lstinline{black}}。另外还有一个自定义的选项  \lstinline{nocolor}。调用颜色主题 \lstinline{green} 的方法为 
% \begin{lstlisting}
% \documentclass[green]{elegantbook} %or
% \documentclass[color=green]{elegantbook}
% \end{lstlisting}


% \begin{table}[htbp]
%   \caption{ElegantBook 模板中的颜色主题\label{tab:color thm}}
%   \centering
%   \begin{tabular}{ccccccc}
%   \toprule
%     & \textcolor{structure1}{green} 
%     & \textcolor{structure2}{cyan} 
%     & \textcolor{structure3}{blue}
%     & \textcolor{structure4}{gray} 
%     & \textcolor{structure5}{black} 
%     & 主要使用的环境\\
%   \midrule
%     structure & \ccr{structure1}
%     & \ccr{structure2}
%     & \ccr{structure3} 
%     & \ccr{structure4} 
%     & \ccr{structure5} 
%     & chapter \ section \ subsection \\
%     main      & \ccr{main1}
%     & \ccr{main2}
%     & \ccr{main3}
%     & \ccr{main4}
%     & \ccr{main5}
%     & definition \ exercise \ problem \\
%     second    & \ccr{second1}
%     & \ccr{second2}
%     & \ccr{second3}
%     & \ccr{second4}
%     & \ccr{second5}
%     & theorem \ lemma \ corollary\\
%     third     & \ccr{third1}
%     & \ccr{third2}
%     & \ccr{third3}
%     & \ccr{third4}
%     & \ccr{third5}
%     & proposition\\
%   \bottomrule
%   \end{tabular}
% \end{table}

% 如果需要自定义颜色的话请选择 \lstinline{nocolor} 选项或者使用 \lstinline{color=none},然后在导言区定义 structurecolor、main、second、third 颜色,具体方法如下:
% \begin{lstlisting}[tabsize=4]
% \definecolor{structurecolor}{RGB}{0,0,0}
% \definecolor{main}{RGB}{70,70,70}    
% \definecolor{second}{RGB}{115,45,2}    
% \definecolor{third}{RGB}{0,80,80}
% \end{lstlisting}

% \section{封面}

% \subsection{封面个性化}

% 从 3.10 版本开始,封面更加弹性化,用户可以自行选择输出的内容,包括 \lstinline{\title} 在内的所有封面元素都可为空。目前封面的元素有

% \begin{table}[htbp]
%   \centering
%   \caption{封面元素信息}
%   \begin{tabular}{p{0.07\textwidth}p{0.15\textwidth}|p{0.07\textwidth}p{0.15\textwidth}|p{0.07\textwidth}p{0.15\textwidth}}
%     \toprule
%     信息 & 命令 & 信息 & 命令 & 信息 & 命令 \\
%     \midrule
%     标题 & \lstinline|\title| & 副标题 & \lstinline|\subtitle| & 作者 & \lstinline|\author| \\
%     机构 & \lstinline|\institute| & 日期 &  \lstinline|\date| & 版本 & \lstinline|\version| \\
%     箴言 & \lstinline|\extrainfo| & 封面图 & \lstinline|\cover| & 徽标 & \lstinline|\logo| \\
%     \bottomrule
%   \end{tabular}
% \end{table}

% 另外,额外增加一个 \lstinline{\bioinfo} 命令,有两个选项,分别是信息标题以及信息内容。比如需要显示{\kaishu User Name:111520},则可以使用 
% \begin{lstlisting}
% \bioinfo{User Name}{115520}
% \end{lstlisting}

% 封面中间位置的色块的颜色可以使用下面命令进行修改:
% \begin{lstlisting}
% \definecolor{customcolor}{RGB}{32,178,170}
% \colorlet{coverlinecolor}{customcolor}
% \end{lstlisting}

% \subsection{封面图}

% 本模板使用的封面图片来源于 \href{https://pixabay.com/en/tea-time-poetry-coffee-reading-3240766/}{pixabay.com}\footnote{感谢 China\TeX{} 提供免费图源网站,另外还推荐 \href{https://www.pexels.com/}{pexels.com}。},图片完全免费,可用于任何场景。封面图片的尺寸为 $1280 \times 1024$, 更换图片的时候请\textbf{严格}按照封面图片尺寸进行裁剪。推荐一个免费的在线图片裁剪网站 \href{https://www.fotor.com/cn}{fotor.com}。用户 QQ 群内有一些合适尺寸的封面,欢迎取用。

% \subsection{徽标}

% 本文用到的 Logo 比例为 1:1,也即正方形图片,在更换图片的时候请选择合适的图片进行替换。

% \subsection{自定义封面}

% 另外,如果使用自定义的封面,比如 Adobe illustrator 或者其他软件制作的 A4 PDF 文档,请把 \lstinline{\maketitle} 注释掉,然后借助 \lstinline{pdfpages} 宏包将自制封面插入即可。如果使用 \lstinline{titlepage} 环境,也是类似。如果需要 2.x 版本的封面,请参考 \href{https://github.com/EthanDeng/etitlepage}{etitlepage}。

% \section{章标标题}

% 本模板内置 2 套\textit{章标题显示风格},包含 \lstinline{hang}(默认)与 \lstinline{display} 两种风格,区别在于章标题单行显示(\lstinline{hang})与双行显示(\lstinline{display}),本说明使用了 \lstinline{hang}。调用方式为
% \begin{lstlisting}
% \documentclass[hang]{elegantbook} %or
% \documentclass[titlestyle=hang]{elegantbook}
% \end{lstlisting}

% 在章标题内,章节编号默认是以数字显示,也即{\kaishu 第 1 章},{\kaishu 第 2 章}等等,如果想要把数字改为中文,可以使用
% \begin{lstlisting}
% \documentclass[chinese]{elegantbook} %or
% \documentclass[scheme=chinese]{elegantbook}
% \end{lstlisting}

% \section{数学环境简介}

% 在我们这个模板中,我们定义了两种不同的定理模式 \lstinline{mode},包括简单模式(\lstinline{simple})和炫彩模式(\lstinline{fancy}),默认为 \lstinline{fancy} 模式,不同模式的选择为
% \begin{lstlisting}
% \documentclass[simple]{elegantbook} %or
% \documentclass[mode=simple]{elegantbook}
% \end{lstlisting}

% 本模板定义了四大类环境

% \begin{itemize}
%   \item \textit{定理类环境},包含标题和内容两部分,全部定理类环境的编号均以章节编号。根据格式的不同分为 3 种
%     \begin{itemize}
%       \item \textcolor{main}{\textbf{definition}} 环境,颜色为 \textcolor{main}{main};
%       \item \textcolor{second}{\textbf{theorem、lemma、corollary、axiom、postulate}} 环境,颜色为 \textcolor{second} {second};
%       \item \textcolor{third}{\textbf{proposition}} 环境,颜色为 \textcolor{third}{third}。
%     \end{itemize}
%   \item \textit{示例类环境},有 \textbf{example、problem、exercise} 环境(对应于例、例题、练习),自动编号,编号以章节为单位,其中 \textbf{exercise} 有提示符。
%   \item \textit{提示类环境},有 \textbf{note} 环境,特点是:无编号,有引导符。
%   \item \textit{结论类环境},有 \textbf{conclusion、assumption、property、remark、solution} 环境\footnote{本模板还添加了一个 result 选项,用于隐藏 \lstinline{solution} 和 \lstinline{proof} 环境,默认为显示(\lstinline{result=answer}),隐藏使用 \lstinline{result=noanswer}。},三者均以粗体的引导词为开头,和普通段落格式一致。
% \end{itemize}

% \subsection{定理类环境的使用}

% 由于本模板使用了 \lstinline{tcolorbox} 宏包来定制定理类环境,所以和普通的定理环境的使用有些许区别,定理的使用方法如下:
% \begin{lstlisting}
% \begin{theorem}{theorem name}{label}
%   The content of theorem.
% \end{theorem}
% \end{lstlisting}

% 第一个必选项 \lstinline{theorem name} 是定理的名字,第二个必选项 \lstinline{label} 是交叉引用时所用到的标签,交叉引用的方法为 \verb|\ref{thm:label}|。请注意,交叉引用时必须加上前缀 \lstinline{thm:}。

% 在用户多次反馈下,4.x 之后,引入了原生定理的支持方式,也就是使用可选项方式:

% \begin{lstlisting}
% \begin{theorem}[theorem name] \label{thm:theorem-label}
%   The content of theorem.
% \end{theorem}
% % or 
% \begin{theorem} \label{thm:theorem-withou-name}
%   The content of theorem without name.
% \end{theorem}
% \end{lstlisting}

% 其他相同用法的定理类环境有:

% \begin{table}[htbp]
%    \centering
%    \caption{定理类环境}
%      \begin{tabular}{llll}
%      \toprule
%      环境名 & 标签名 & 前缀 & 交叉引用 \\
%      \midrule
%      definition & label & def   & \lstinline|\ref{def:label}| \\
%      theorem & label & thm   & \lstinline|\ref{thm:label}| \\
%      postulate & label & pos & \lstinline|\ref{pos:label}| \\
%      axiom & label & axi & \lstinline|\ref{axi:label}|\\
%      lemma & label & lem   & \lstinline|\ref{lem:label}| \\
%      corollary & label & cor   & \lstinline|\ref{cor:label}| \\
%      proposition & label & pro   & \lstinline|\ref{pro:label}| \\
%      \bottomrule
%      \end{tabular}%
%    \label{tab:theorem-class}%
%  \end{table}%
 
% % \subsection{算法环境}

 
% % \begin{algorithm}\label{alg:test}
% %   \Input{A bitmap $I$ of size $w \times l$}
% %   \Output{A partition of the bitmap}
% %   \BlankLine
% %   \emph{special treatment of the first line}\;
% %   \For{$i \leftarrow 2$ \KwTo $l$}{
% %     \emph{special treatment of the first element of line $i$}\;
% %     \For{$j \leftarrow 2$ \KwTo $w$}{\label{forins}
% %       $\Left \leftarrow \FindCompress{$I[i,j-1]$}$\;
% %       $\Up \leftarrow \FindCompress{$I[i-1,]$}$\;
% %       $\This \leftarrow \FindCompress{$I[i,j]$}$\;
% %       \If(\tcp*[h]{O(\Left,\This)==1}){\Left compatible with \This}{\label{lt}
% %         \lIf{$\Left < \This$}{$\Union{\Left,\This}$}
% %         \lElse{$\Union{\This,\Left}$}
% %       }
% %       \If(\tcp*[f]{O(\Up,\This)==1}){\Up compatible with \This}{\label{ut}
% %         \lIf{$\Up < \This$}{$\Union{\Up,\This}$}
% %         \tcp{\This is put under \Up to keep tree as flat as possible}\label{cmt}
% %         \lElse{$\Union{\This,\Up}$}\tcp*[r]{\This{} linked to \Up}\label{lelse}
% %       }
% %     }
% %     \lForEach{element $e$ of the line $i$}{\FindCompress{p}}
% %   }
% %   \caption{disjoint decomposition}\label{algo_disjdecomp}
% % \end{algorithm}


% \subsection{修改计数器}

% 当前定理等环境计数器按章计数,如果想修改定理类环境按节计数,可以修改计数器选项 thmcnt:

% \begin{lstlisting}
%   \documentclass[section]{elegantbook} %or
%   \documentclass[thmcnt=section]{elegantbook}
% \end{lstlisting}


% \subsection{其他环境的使用}

% 其他三种环境没有选项,可以直接使用,比如 \lstinline{example} 环境的使用方法与效果:
% \begin{lstlisting}
% \begin{example}
%    This is the content of example environment.
% \end{example}
% \end{lstlisting}

% 这几个都是同一类环境,区别在于

% \begin{itemize}
%   \item 示例环境(example)、练习(exercise)与例题(problem)章节自动编号;
%   \item 注意(note),练习(exercise)环境有提醒引导符;
%   \item 结论(conclusion)等环境都是普通段落环境,引导词加粗。
% \end{itemize}

% \section{列表环境}
% 本模板借助于 \lstinline{tikz} 定制了 \lstinline{itemize} 和 \lstinline{enumerate} 环境,其中 \lstinline{itemize} 环境修改了 3 层嵌套,而 \lstinline{enumerate} 环境修改了 4 层嵌套(仅改变颜色)。示例如下\\[2ex]
% \begin{minipage}[b]{0.49\textwidth}
%   \begin{itemize}
%     \item first item of nesti;
%     \item second item of nesti;
%       \begin{itemize}
%         \item first item of nestii;
%         \item second item of nestii;
%         \begin{itemize}
%           \item first item of nestiii;
%           \item second item of nestiii.
%         \end{itemize}   
%       \end{itemize}
%   \end{itemize}
% \end{minipage}
% \begin{minipage}[b]{0.49\textwidth}
%   \begin{enumerate}
%     \item first item of nesti;
%     \item second item of nesti;
%       \begin{enumerate}
%         \item first item of nestii;
%         \item second item of nestii;
%         \begin{enumerate}
%           \item first item of nestiii;
%           \item second item of nestiii.
%         \end{enumerate}   
%       \end{enumerate}
%   \end{enumerate}
% \end{minipage}

% \section{参考文献}
% 文献部分,本模板调用了 biblatex 宏包,并提供了 biber(默认) 和 bibtex 两个后端选项,可以使用 \lstinline{bibend} 进行修改:

% \begin{lstlisting}
%   \documentclass[bibtex]{elegantbook}
%   \documentclass[bibend=bibtex]{elegantbook}
% \end{lstlisting}

% 关于文献条目(bib item),你可以在谷歌学术,Mendeley,Endnote 中取,然后把它们添加到 \lstinline{reference.bib} 中。在文中引用的时候,引用它们的键值(bib key)即可。

% 为了方便文献样式修改,模板引入了 \lstinline{bibstyle} 和 \lstinline{citestyle} 选项,默认均为数字格式(numeric),参考文献示例:\cite{cn1,en2,en3} 使用了中国一个大型的 P2P 平台(人人贷)的数据来检验男性投资者和女性投资者在投资表现上是否有显著差异。

% 如果需要设置为国标 GB7714-2015,需要使用:
% \begin{lstlisting}
%   \documentclass[citestyle=gb7714-2015, bibstyle=gb7714-2015]{elegantbook} 
% \end{lstlisting}

% 如果需要添加排序方式,可以在导言区加入
% \begin{lstlisting}
%   \ExecuteBibliographyOptions{sorting=ynt}
% \end{lstlisting}

% 启用国标之后,可以加入 \lstinline{sorting=gb7714-2015}。

% \section{添加序章}

% 如果你想在第一章前面添序章,不改变原本章节序号,可以在第一章内容前面使用 
% \begin{lstlisting}
% \chapter*{Introduction}
% \markboth{Introduction}{Introduction}
% The content of introduction.
% \end{lstlisting}

% \section{目录选项与深度}
% 本模板添加了一个目录选项 \lstinline{toc},可以设置目录为单栏(\lstinline{onecol})和双栏(\lstinline{twocol})显示,比如双栏显示可以使用
% \begin{lstlisting}
% \documentclass[twocol]{elegantbook}
% \documentclass[toc=twocol]{elegantbook}
% \end{lstlisting}

% 默认本模板目录深度为 1,你可以在导言区使用
% \begin{lstlisting}
% \setcounter{tocdepth}{2}
% \end{lstlisting}
% 将其修改为 2 级目录(章与节)显示。


% \section{章节摘要}
% 模板新增了一个章节摘要环境(introduction),使用示例
% \begin{lstlisting}
% \begin{introduction}
%   \item Definition of Theorem
%   \item Ask for help
%   \item Optimization Problem
%   \item Property of Cauchy Series
%   \item Angle of Corner
% \end{introduction}
% \end{lstlisting}
% 效果如下:
% \begin{introduction}
%   \item Definition of Theorem
%   \item Ask for help
%   \item Optimization Problem
%   \item Property of Cauchy Series
%   \item Angle of Corner
% \end{introduction}

% 环境的标题文字可以通过这个环境的可选参数进行修改,修改方法为:
% \begin{lstlisting}
% \begin{introduction}[Brief Introduction]
% ...
% \end{introduction}
% \end{lstlisting}

% \section{章后习题}
% 前面我们介绍了例题和练习两个环境,这里我们再加一个,章后习题(\lstinline{problemset})环境,用于在每一章结尾,显示本章的练习。使用方法如下

% \begin{lstlisting}
% \begin{problemset}
%   \item exercise 1
%   \item exercise 2
%   \item exercise 3
% \end{problemset}
% \end{lstlisting}


% 效果如下:
% \begin{problemset}
%   \item exercise 1
%   \item exercise 2
%   \item exercise 3
%   \item 测试数学公式
%   \begin{equation}
%     a^2+b^2=c_{2_{i}} (1,2) [1,23]
%   \end{equation}
% \end{problemset}

% \begin{remark}
% 如果你想把 \lstinline{problemset} 环境的标题改为其他文字,你可以类似于 introduction 环境修改 problemset 的可选参数。另外,目前这个环境会自动出现在目录中,但是不会出现在页眉页脚信息中(待解决)。
% \end{remark}

% \begin{solution}
% 如果你想把 \lstinline{problemset} 环境的标题改为其他文字,你可以类似于 introduction 环境修改 problemset 的可选参数。另外,目前这个环境会自动出现在目录中,但是不会出现在页眉页脚信息中(待解决)。
% \end{solution}

% \section{旁注}

% 在 3.08 版本中,我们引入了 旁注设置选项 \lstinline{marginpar=margintrue} 以及测试命令 \lstinline{\elegantpar} ,但是由此带来一堆问题。我们决定在 3.09 版本中将其删除,并且,在旁注命令得到大幅度优化之前,不会将此命令再次引入书籍模板中。对此造成各位用户的不方便,非常抱歉!不过我们保留了 \lstinline{marginpar} 这个选项,你可以使用 \lstinline{marginpar=margintrue} 获得保留右侧旁注的版面设计。然后使用系统自带的 \lstinline{\marginpar} 或者 \lstinline{marginnote} 宏包的 \lstinline{\marginnote} 命令。

% \begin{remark}
% 在使用旁注的时候,需要注意的是,文本和公式可以直接在旁注中使用。

% \begin{lstlisting}
% % text
% \marginpar{margin paragraph text}

% % equation
% \marginpar{
%   \begin{equation}
%     a^2 + b^2 = c^2
%   \end{equation}
% }
% \end{lstlisting}

% 但是浮动体(表格、图片)需要注意,不能用浮动体环境,需要使用直接插图命令或者表格命令环境。然后使用 \lstinline{\captionof} 为其设置标题。为了得到居中的图表,可以使用 \lstinline{\centerline} 命令或者 \lstinline{center} 环境。更多详情请参考:\href{https://tex.stackexchange.com/questions/5583/caption-of-figure-in-marginpar-and-caption-of-wrapfigure-in-margin}{Caption of Figure in Marginpar}。

% \begin{lstlisting}
% % graph with centerline command
% \marginpar{
%   \centerline{
%     \includegraphics[width=0.2\textwidth]{logo.png}
%   }
%   \captionof{figure}{your figure caption}
% }

% % graph with center environment
% \marginpar{
%   \begin{center}
%     \includegraphics[width=0.2\textwidth]{logo.png}
%     \captionof{figure}{your figure caption}
%   \end{center}
% }
% \end{lstlisting}

% \end{remark}

% \chapter{字体选项}
% 字体选项独立成章的原因是,我们希望本模板的用户关心模板使用的字体,知晓自己使用的字体以及遇到字体相关的问题能更加便捷地找到答案。

% \textcolor{red}{\bfseries 重要提示}:从 3.10 版本更新之后,沿用至今的 newtx 系列字体被重新更改为 cm 字体。并且新增中文字体(\lstinline{chinesefont})选项。

% \section{数学字体选项}

% 本模板定义了一个数学字体选项(\lstinline{math}),可选项有三个:
% \begin{enumerate}
%   \item \lstinline{math=cm}(默认),使用 \LaTeX{} 默认数学字体(推荐,无需声明);
%   \item \lstinline{math=newtx},使用 \lstinline{newtxmath} 设置数学字体(潜在问题比较多)。
%   \item \lstinline{math=mtpro2},使用 \lstinline{mtpro2} 宏包设置数学字体,要求用户已经成功安装此宏包。
% \end{enumerate}

% \section{使用 newtx 系列字体}

% 如果需要使用原先版本的 \lstinline{newtx} 系列字体,可以通过显示声明数学字体:

% \begin{lstlisting}
% \documentclass[math=newtx]{elegantbook}
% \end{lstlisting}

% \subsection{连字符}

% 如果使用 \lstinline{newtx} 系列字体宏包,需要注意下连字符的问题。
% \begin{equation}
%   \int_{R^q} f(x,y) dy.\emph{of\kern0pt f}
% \end{equation}
% 的代码为
% \begin{lstlisting}
% \begin{equation}
%   \int_{R^q} f(x,y) dy.\emph{of \kern0pt f}
% \end{equation}
% \end{lstlisting}

% \subsection{宏包冲突}

% 另外在 3.08 版本中,有用户反馈模板在和 \lstinline{yhmath} 以及 \lstinline{esvect} 等宏包搭配使用的时候会出现报错:
% \begin{lstlisting}
% LaTeX Error:
%    Too many symbol fonts declared.
% \end{lstlisting}

% 原因是在使用 \lstinline{newtxmath} 宏包时,重新定义了数学字体用于大型操作符,达到了 {\heiti 最多 16 个数学字体} 的上限,在调用其他宏包的时候,无法新增数学字体。为了减少调用非常用宏包,在此给出如何调用 \lstinline{yhmath} 以及 \lstinline{esvect} 宏包的方法。

% 请在 \lstinline{elegantbook.cls} 内搜索 \lstinline{yhmath} 或者 \lstinline{esvect},将你所需要的宏包加载语句\textit{取消注释}即可。
% \begin{lstlisting}
% %%% use yhmath pkg, uncomment following code
% % \let\oldwidering\widering
% % \let\widering\undefined
% % \RequirePackage{yhmath}
% % \let\widering\oldwidering

% %%% use esvect pkg, uncomment following code
% % \RequirePackage{esvect}
% \end{lstlisting}

% \section{中文字体选项}
% 模板从 3.10 版本提供中文字体选项 \lstinline{chinesefont},可选项有
% \begin{enumerate}
% \item \lstinline{ctexfont}:默认选项,使用 \lstinline{ctex} 宏包根据系统自行选择字体,可能存在字体缺失的问题,更多内容参考 \lstinline{ctex} 宏包\href{https://ctan.org/pkg/ctex}{官方文档}\footnote{可以使用命令提示符,输入 \lstinline{texdoc ctex} 调出本地 \lstinline{ctex} 宏包文档}。
% \item \lstinline{founder}:方正字体选项(\textbf{需要安装方正字体}),后台调用 \lstinline{ctex} 宏包并且使用 \lstinline{fontset=none} 选项,然后设置字体为方正四款免费字体,方正字体下载注意事项见后文,用户只需要安装方正字体即可使用该选项。
% \item \lstinline{nofont}:后台会调用 \lstinline{ctex} 宏包并且使用 \lstinline{fontset=none} 选项,不设定中文字体,用户可以自行设置中文字体,具体见后文。
% \end{enumerate}

% \subsection{方正字体选项}
% 由于使用 \lstinline{ctex} 宏包默认调用系统已有的字体,部分系统字体缺失严重,因此,用户希望能够使用其它字体,我们推荐使用方正字体。方正的{\songti 方正书宋}、{\heiti 方正黑体}、{\kaishu 方正楷体}、{\fangsong 方正仿宋}四款字体均可免费试用,且可用于商业用途。用户可以自行从\href{http://www.foundertype.com/}{方正字体官网}下载此四款字体,在下载的时候请\textbf{务必}注意选择 GBK 字符集,也可以使用 \href{https://www.latexstudio.net/}{\LaTeX{} 工作室}提供的\href{https://pan.baidu.com/s/1BgbQM7LoinY7m8yeP25Y7Q}{方正字体,提取码为:njy9} 进行安装。安装时,{\kaishu Win 10 用户请右键选择为全部用户安装,否则会找不到字体。}

% \begin{figure}[!htb]
% \centering
% \includegraphics[width=0.9\textwidth]{founder.png}
% \end{figure}

% \subsection{其他中文字体}
% 如果你想完全自定义字体\footnote{这里仍然以方正字体为例。},你可以选择 \lstinline{chinesefont=nofont},然后在导言区设置
% \begin{lstlisting}
% \setCJKmainfont[BoldFont={FZHei-B01},ItalicFont={FZKai-Z03}]{FZShuSong-Z01}
% \setCJKsansfont[BoldFont={FZHei-B01}]{FZKai-Z03}
% \setCJKmonofont[BoldFont={FZHei-B01}]{FZFangSong-Z02}
% \setCJKfamilyfont{zhsong}{FZShuSong-Z01}
% \setCJKfamilyfont{zhhei}{FZHei-B01}
% \setCJKfamilyfont{zhkai}[BoldFont={FZHei-B01}]{FZKai-Z03}
% \setCJKfamilyfont{zhfs}[BoldFont={FZHei-B01}]{FZFangSong-Z02}
% \newcommand*{\songti}{\CJKfamily{zhsong}}
% \newcommand*{\heiti}{\CJKfamily{zhhei}}
% \newcommand*{\kaishu}{\CJKfamily{zhkai}}
% \newcommand*{\fangsong}{\CJKfamily{zhfs}}
% \end{lstlisting}

% \chapter{ElegantBook 写作示例}

% \begin{introduction}
%   \item 积分定义~\ref{def:int}
%   \item Fubini 定理~\ref{thm:fubi}
%   \item 最优性原理~\ref{pro:max}
%   \item 柯西列性质~\ref{property:cauchy}
%   \item 韦达定理
% \end{introduction}

% \section{Lebesgue 积分}
% 在前面各章做了必要的准备后,本章开始介绍新的积分。在 Lebesgue 测度理论的基础上建立了 Lebesgue 积分,其被积函数和积分域更一般,可以对有界函数和无界函数统一处理。正是由于 Lebesgue 积分的这些特点,使得 Lebesgue 积分比 Riemann 积分具有在更一般条件下的极限定理和累次积分交换积分顺序的定理,这使得 Lebesgue 积分不仅在理论上更完善,而且在计算上更灵活有效。

% Lebesgue 积分有几种不同的定义方式。我们将采用逐步定义非负简单函数,非负可测函数和一般可测函数积分的方式。

% 由于现代数学的许多分支如概率论、泛函分析、调和分析等常常用到一般空间上的测度与积分理论,在本章最后一节将介绍一般的测度空间上的积分。

% \subsection{积分的定义}

% 我们将通过三个步骤定义可测函数的积分。首先定义非负简单函数的积分。以下设 $E$ 是 $\mathcal{R}^n$ 中的可测集。

% \begin{definition}[可积性] \label{def:int} 
% 设 $ f(x)=\sum\limits_{i=1}^{k} a_i \chi_{A_i}(x)$ 是 $E$ 上的\textbf{非负简单函数},中文其中 $\{A_1,A_2,\ldots,A_k\}$ 是 $E$ 上的一个可测分割,$a_1,a_2,\ldots,a_k$ 是非负实数。定义 $f$ 在 $E$ 上的积分为 $\int_{a}^b f(x)$
% \begin{equation}
%    \label{inter}
%    \int_{E} f dx = \sum_{i=1}^k a_i m(A_i) \pi \alpha\beta\sigma\gamma\nu\xi\epsilon\varepsilon. \oint_{a}^b\ointop_{a}^b\prod_{i=1}^n
% \end{equation}
% 一般情况下 $0 \leq \int_{E} f dx \leq \infty$。若 $\int_{E} f dx < \infty$,则称 $f$ 在 $E$ 上可积。
% \end{definition}

% 一个自然的问题是,Lebesgue 积分与我们所熟悉的 Riemann 积分有什么联系和区别?在 4.4 在我们将详细讨论 Riemann 积分与 Lebesgue 积分的关系。这里只看一个简单的例子。设 $D(x)$ 是区间 $[0,1]$ 上的 Dirichlet 函数。即 $D(x)=\chi_{Q_0}(x)$,其中 $Q_0$ 表示 $[0,1]$ 中的有理数的全体。根据非负简单函数积分的定义,$D(x)$ 在 $[0,1]$ 上的 Lebesgue 积分为
% \begin{equation}
%    \label{inter2}
%    \int_0^1 D(x)dx = \int_0^1 \chi_{Q_0} (x) dx = m(Q_0) = 0
% \end{equation}
% 即 $D(x)$ 在 $[0,1]$ 上是 Lebesgue 可积的并且积分值为零。但 $D(x)$ 在 $[0,1]$ 上不是 Riemann 可积的。


% 有界变差函数是与单调函数有密切联系的一类函数。有界变差函数可以表示为两个单调递增函数之差。与单调函数一样,有界变差函数几乎处处可导。与单调函数不同,有界变差函数类对线性运算是封闭的,它们构成一线空间。练习题 \ref{exer:43} 是一个性质的证明。

% \begin{exercise}\label{exer:43}
% 设 $f \notin\in L(\mathcal{R}^1)$,$g$ 是 $\mathcal{R}^1$ 上的有界可测函数。证明函数
% \begin{equation}
%    \label{ex:1}
%    I(t) = \int_{\mathcal{R}^1} f(x+t)g(x)dx \quad t \in \mathcal{R}^1
% \end{equation}
% 是 $\mathcal{R}^1$ 上的连续函数。 
% \end{exercise}

% \begin{solution}
% 即 $D(x)$ 在 $[0,1]$ 上是 Lebesgue 可积的并且积分值为零。但 $D(x)$ 在 $[0,1]$ 上不是 Riemann 可积的。
% \end{solution}

% \begin{proof}
% 即 $D(x)$ 在 $[0,1]$ 上是 Lebesgue 可积的并且积分值为零。但 $D(x)$ 在 $[0,1]$ 上不是 Riemann 可积的。
% \end{proof}

% \begin{theorem}[Fubini 定理] \label{thm:fubi} 
% (1)若 $f(x,y)$ 是 $\mathcal{R}^p\times\mathcal{R}^q$ 上的非负可测函数,则对几乎处处的 $x\in \mathcal{R}^p$,$f(x,y)$ 作为 $y$ 的函数是 $\mathcal{R}^q$ 上的非负可测函数,$g(x)=\int_{\mathcal{R}^q}f(x,y) dy$ 是 $\mathcal{R}^p$ 上的非负可测函数。并且
% \begin{equation}
%    \label{eq:461}
%    \int_{\mathcal{R}^p\times\mathcal{R}^q} f(x,y) dxdy=\int_{\mathcal{R}^p}\left(\int_{\mathcal{R}^q}f(x,y)dy\right)dx.
% \end{equation}

% (2)若 $f(x,y)$ 是 $\mathcal{R}^p\times\mathcal{R}^q$ 上的可积函数,则对几乎处处的 $x\in\mathcal{R}^p$,$f(x,y)$ 作为 $y$ 的函数是 $\mathcal{R}^q$ 上的可积函数,并且 $g(x)=\int_{\mathcal{R}^q}f(x,y) dy$ 是 $\mathcal{R}^p$ 上的可积函数。而且~\ref{eq:461} 成立。
% \end{theorem}

% \ref{thm:fubi}

% \begin{note}
% 在本模板中,引理(lemma),推论(corollary)的样式和定理~\ref{thm:fubi} 的样式一致,包括颜色,仅仅只有计数器的设置不一样。
% \end{note}

% 我们说一个实变或者复变量的实值或者复值函数是在区间上平方可积的,如果其绝对值的平方在该区间上的积分是有限的。所有在勒贝格积分意义下平方可积的可测函数构成一个希尔伯特空间,也就是所谓的 $L^2$ 空间,几乎处处相等的函数归为同一等价类。形式上,$L^2$ 是平方可积函数的空间和几乎处处为 0 的函数空间的商空间。

% \begin{proposition}[最优性原理] \label{pro:max}
% 如果 $u^*$ 在 $[s,T]$ 上为最优解,则 $u^*$ 在 $[s, T]$ 任意子区间都是最优解,假设区间为 $[t_0, t_1]$ 的最优解为 $u^*$ ,则 $u(t_0)=u^{*}(t_0)$,即初始条件必须还是在 $u^*$ 上。
% \end{proposition}

% 我们知道最小二乘法可以用来处理一组数据,可以从一组测定的数据中寻求变量之间的依赖关系,这种函数关系称为经验公式。本课题将介绍最小二乘法的精确定义及如何寻求点与点之间近似成线性关系时的经验公式。假定实验测得变量之间的 $n$ 个数据,则在平面上,可以得到 $n$ 个点,这种图形称为 “散点图”,从图中可以粗略看出这些点大致散落在某直线近旁, 我们认为其近似为一线性函数,下面介绍求解步骤。

% \begin{figure}[htbp]
%   \centering
%   \includegraphics[width=0.6\textwidth]{scatter.jpg}
%   \caption{散点图示例 $\hat{y}=a+bx$ \label{fig:scatter}}
% \end{figure}

% 以最简单的一元线性模型来解释最小二乘法。什么是一元线性模型呢?监督学习中,如果预测的变量是离散的,我们称其为分类(如决策树,支持向量机等),如果预测的变量是连续的,我们称其为回归。回归分析中,如果只包括一个自变量和一个因变量,且二者的关系可用一条直线近似表示,这种回归分析称为一元线性回归分析。如果回归分析中包括两个或两个以上的自变量,且因变量和自变量之间是线性关系,则称为多元线性回归分析。对于二维空间线性是一条直线;对于三维空间线性是一个平面,对于多维空间线性是一个超平面。

% \begin{property}\label{property:cauchy}
% 柯西列的性质
% \begin{enumerate}
% \item $\{x_k\}$ 是柯西列,则其子列 $\{x_k^i\}$ 也是柯西列。
% \item $x_k\in \mathcal{R}^n$,$\rho(x,y)$ 是欧几里得空间,则柯西列收敛,$(\mathcal{R}^n,\rho)$ 空间是完备的。
% \end{enumerate}
% \end{property}

% \begin{conclusion}
% 回归分析(regression analysis) 是确定两种或两种以上变量间相互依赖的定量关系的一种统计分析方法。运用十分广泛,回归分析按照涉及的变量的多少,分为一元回归和多元回归分析;按照因变量的多少,可分为简单回归分析和多重回归分析;按照自变量和因变量之间的关系类型,可分为线性回归分析和非线性回归分析。
% \end{conclusion}

% \begin{problemset}
% \item 设 $A$ 为数域 $K$ 上的 $n$ 级矩阵。证明:如果 $K^n$ 中任意非零列向量都是 $A$ 的特征向量,则 $A$ 一定是数量矩阵。
% \item 证明:不为零矩阵的幂零矩阵不能对角化。
% \item 设 $A = (a_{ij})$ 是数域 $K$ 上的一个 $n$ 级上三角矩阵,证明:如果 $a_{11} = a_{22} = \cdots = a_{nn}$,并且至少有一个 $a_{kl} \not = 0 (k < l)$,则 $A$ 一定不能对角化。
% \end{problemset}

% \chapter{常见问题集}

% 我们根据用户社区反馈整理了下面一些常见的问题,用户在遇到问题时,应当首先查阅本手册和本部分的常见的问题。

% \begin{enumerate}[itemsep=1.5ex]
%   \item \question{有没有办法章节用“第一章,第一节,(一)”这种?}
%     见前文介绍,可以使用 \lstinline{scheme=chinese} 设置。
%   \item \question{大佬,我想把正文字体改为亮色,背景色改为黑灰色。}
%     页面颜色可以使用 \lstinline{\pagecolor} 命令设置,文本命令可以参考\href{https://tex.stackexchange.com/questions/278544/xcolor-what-is-the-equivalent-of-default-text-color}{这里}进行设置。
%   \item \question{\lstinline[breaklines]{Package ctex Error: CTeX fontset 'Mac' is unavailable.}}
%     在 Mac 系统下,中文编译请使用 \hologo{XeLaTeX}。
%   \item \question{\lstinline{! LaTeX Error: Unknown option 'scheme=plain' for package 'ctex'.}}
%     你用的 C\TeX{} 套装吧?这个里面的 \lstinline{ctex} 宏包已经是已经是 10 年前的了,与本模板使用的 \lstinline{ctex} 宏集有很大区别。不建议 C\TeX{} 套装了,请卸载并安装 \TeX{} Live 2022。
%   \item \question{我该使用什么版本?}
%     请务必使用\href{https://github.com/ElegantLaTeX/ElegantBook/releases}{最新正式发行版},发行版间不定期可能会有更新(修复 bug 或者改进之类),如果你在使用过程中没有遇到问题,不需要每次更新\href{https://github.com/ElegantLaTeX/ElegantBook/archive/master.zip}{最新版},但是在发行版更新之后,请尽可能使用最新版(发行版)!最新发行版可以在 GitHub 或者 \TeX{} Live 2021 内获取。
%   \item \question{我该使用什么编辑器?}
%     你可以使用 \TeX{} Live 2021 自带的编辑器 \TeX{}works 或者使用 \TeX{}studio,\TeX works 的自动补全,你可以参考我们的总结 \href{https://github.com/EthanDeng/texworks-autocomplete}{\TeX works 自动补全}。推荐使用 \TeX{} Live 2021 + \TeX{}studio。我自己用 VS Code 和 Sublime Text,相关的配置说明,请参考 \href{https://github.com/EthanDeng/vscode-latex}{\LaTeX{} 编译环境配置:Visual Studio Code 配置简介} 和 \href{https://github.com/EthanDeng/sublime-text-latex}{Sublime Text 搭建 \LaTeX{} 编写环境}。
%   \item \question{您好,我们想用您的 ElegantBook 模板写一本书。关于机器学习的教材,希望获得您的授权,谢谢您的宝贵时间。}
%     模板的使用修改都是自由的,你们声明模板来源以及模板地址(GitHub 地址)即可,其他未尽事宜按照开源协议 LPPL-1.3c。做好之后,如果方便的话,可以给我们一个链接,我把你们的教材放在 Elegant\LaTeX{} 用户作品集里。
%   \item \question{请问交叉引用是什么?}
%     本群和本模板适合有一定 \LaTeX{} 基础的用户使用,新手请先学习 \LaTeX{} 的基础,理解各种概念,否则你将寸步难行。
%   \item \question{代码高亮环境能用其他语言吗?}
%     可以的,ElegantBook 模板用的是 \lstinline{listings} 宏包,你可以在环境(\lstinline{lstlisting})之后加上语言(比如 Python 使用 \lstinline{language=Python} 选项),全局语言修改请使用 \lstinline{lsset} 命令,更多信息请参考宏包文档。
%   \item \question{群主,什么时候出 Beamer 的模板(主题),ElegantSlide 或者 ElegantBeamer?}
%     由于 Beamer 中有一个很优秀的主题 \href{https://github.com/matze/mtheme}{Metropolis}。后续确定不会再出任何主题/模板,请大家根据需要修改已有主题。
% \end{enumerate}

% \chapter{版本更新历史}

% 根据用户的反馈,我们不断修正和完善模板。由于 3.00 之前版本与现在版本差异非常大,在此不列出 3.00 之前的更新内容。


% \datechange{2022/04/09}{版本 4.3 正式发布。}

% \begin{change}
%   \item 放弃 newtx 系列宏包的设置,改用 TeX Gyre Terms,并设置其他字体;
%   \item 修改定理类环境内部字体设置,修复环境内部中文无法加粗问题;
%   \item 增加定理类环境的计数器选项 \lstinline{thmcnt},可选 \lstinline{chapter} 和 \lstinline{section};
%   \item 增加 \lstinline{bibend} 选项,可选 \lstinline{bibend=biber}(默认)和 \lstinline{bibend=bibtex}。
% \end{change}



% \datechange{2022/03/08}{版本 4.2 正式发布。}

% \begin{change}
%   \item 对于 newtx 系列宏包更新导致的字体 bug 的修复;
%   \item 修缮目录格式,为了达到这个目的,重新改写 \lstinline{\chaptername} 的重定义语句;
%   \item 增加日语 \lstinline{lang=jp} 设定。
%   \item 这个版本为一个临时性版本,在 \TeX Live 2022 发布之后,将尽快发布 4.3 版本,由于对于中文的改动比较大,可能会出现预期之外的 bug,有问题可以在 QQ 群或者 Github 反馈。
% \end{change}


% \datechange{2021/05/02}{版本 4.1 正式发布。}

% \begin{change}
%   \item \textbf{重要改动}:由原先的 \hologo{BibTeX} 改为 biblatex 编译方式(后端为 \lstinline{biber}),请注意两者之间的差异;
%   \item \textbf{重要改进}:修改对于定理写法兼容方式,提高数学公式代码的兼容性;
%   \item 页面设置改动,默认页面更宽;方便书写和阅读;
%   \item 支持目录文字以及页码跳转;
%   \item 不再维护 \hologo{pdfLaTeX} 中文支持方式,请务必使用 \hologo{XeLaTeX} 编译中文文稿。
%   \item 增加多个语言选项,法语 \lstinline{lang=fr}、荷兰语 \lstinline{lang=nl}、匈牙利语 \lstinline{lang=hu}、西班牙语 \lstinline{lang=es}、蒙古语 \lstinline{lang=mn} 等。
% \end{change}


% \datechange{2020/04/12}{版本 3.11 正式发布,\textcolor{red}{此版本为 3.x 最后版本。}}

% \begin{change}
%   \item \textbf{重要修正}:修复因为 \lstinline{gbt7714} 宏包更新导致的 \lstinline{natbib option clash} 错误;
%   \item 由于 \lstinline{pgfornament} 宏包未被 \TeX{} Live 2020 收录,因此删除 base 相关的内容;
%   \item 修复部分环境的空格问题;
%   \item 增加了意大利语言选项 \lstinline{lang=it}。
% \end{change}


% \datechange{2020/02/10}{版本 3.10 正式发布}

% \begin{change}
%   \item 增加数学字体选项 \lstinline{math},可选项为 \lstinline{newtx} 和 \lstinline{cm}。\\
%   \textbf{重要提示}:原先通过 \lstinline{newtxmath} 宏包设置的数学字体改为 \LaTeX{} 默认数学字体,如果需要保持原来的字体,需要显式声明数学字体(\lstinline{math=newtx});
%   \item 新增中文字体选项 \lstinline{chinesefont},可选项为 \lstinline{ctexfont}、\lstinline{founder} 和 \lstinline{nofont}。
%   \item 将封面作者信息设置为可选,并且增加自定义信息命令 \lstinline{\bioinfo};
%   \item 在说明文档中增加版本历史,新增 \lstinline{\datechange} 命令和 \lstinline{change} 环境;
%   \item 增加汉化章节选项 \lstinline{scheme},可选项为汉化 \lstinline{chinese};
%   \item 由于 \lstinline{\lvert} 问题已经修复,重新调整 \lstinline{ctex} 宏包和 \lstinline{amsmath} 宏包位置。
%   \item 修改页眉设置,去除了 \lstinline{\lastpage} 以避免 page anchor 问题,加入 \lstinline{\frontmatter}。
%   \item 修改参考文献选项 \lstinline{cite},可选项为数字 \lstinline{numbers}、 作者-年份 \lstinline{authoryear} 以及上标 \lstinline{super}。
%   \item 新增参考文献样式选项 \lstinline{bibstyle},并将英文模式下参考文献样式 \lstinline{apalike} 设置为默认值,中文仍然使用 \lstinline{gbt7714} 宏包设置。
% \end{change}

% \datechange{2019/08/18}{版本 3.09 正式发布}

% \begin{change}
%   \item \lstinline{\elegantpar} 存在 bug,删除 \lstinline{\elegantpar} 命令,建议用户改用 \lstinline{\marginnote} 和 \lstinline{\marginpar} 旁注命令。
%   \item 积分操作符统一更改为 \lstinline{esint} 宏包设置;
%   \item 新增目录选项 \lstinline{toc},可选项为单栏 \lstinline{onecol} 和双栏 \lstinline{twocol};
%   \item 手动增加参考文献选项 \lstinline{cite},可选项为上标形式 \lstinline{super};
%   \item 修正章节习题(\lstinline{problemset})环境。
% \end{change}

% \datechange{2019/05/28}{版本 3.08 正式发布}

% \begin{change}
%   \item 修复 \lstinline{\part} 命令。
%   \item 引入 Note 模板中的 \lstinline{pad} 选项 \lstinline{device=pad}。
%   \item 数学字体加入 \lstinline{mtpro2} 可选项 \lstinline{math=mtpro2},使用免费的 \lstinline{lite} 子集。
%   \item 将参考文献默认显示方式 \lstinline{authoyear} 改为 \lstinline{numbers}。
%   \item 引入旁注命令 \lstinline{\marginpar}(测试)。
%   \item 新增章节摘要环境 \lstinline{introduction}。
%   \item 新增章节习题环境 \lstinline{problemset}。
%   \item 将 \lstinline{\equote} 重命名为 \lstinline{\extrainfo}。
%   \item 完善说明文档,增加致谢部分。
% \end{change}

% \datechange{2019/04/15}{版本 3.07 正式发布}

% \begin{change}
%   \item 删除中英文自定义字体总设置。
%   \item 新增颜色主题,并将原绿色默认主题设置为蓝色 \lstinline{color=blue}。
%   \item 引入隐藏装饰图案选项 \lstinline{base},可选项有显示 \lstinline{show} 和隐藏 \lstinline{hide}。
%   \item 新增定理模式 \lstinline{mode},可选项有简单模式 \lstinline{simple} 和炫彩模式 \lstinline{fancy}。
%   \item 新增隐藏证明、答案等环境的选项 \lstinline{result=noanswer}。
% \end{change}

% \datechange{2019/02/25}{版本 3.06 正式发布}

% \begin{change}
%   \item 删除水印。
%   \item 新封面,新装饰图案。
%   \item 添加引言使用说明。
%   \item 修复双面 \lstinline{twoside}。
%   \item 美化列表环境。
%   \item 增加 \lstinline{\subsubsection} 的设置。
%   \item 将模板拆分成中英文语言模式。
%   \item 使用 \lstinline{lstlisting} 添加代码高亮。
%   \item 增加定理类环境使用说明。
% \end{change}

% \datechange{2019/01/22}{版本 3.05 正式发布}

% \begin{change}
%   \item 添加 \lstinline{xeCJK} 宏包中文支持方案。
%   \item 修复模板之前对 Ti\textit{k}Z 单位的改动。
%   \item 更新 logo 图。
% \end{change}

% \datechange{2019/01/15}{版本 3.04 正式发布}

% \begin{change}
%   \item 格式化模板代码。
%   \item 增加 \lstinline{\equote} 命令。
%   \item 修改 \lstinline{\date}。
% \end{change}

% \datechange{2019/01/08}{版本 3.03 正式发布}

% \begin{change}
%   \item 修复附录章节显示问题。
%   \item 小幅优化封面代码。
% \end{change}

% \datechange{2018/12/31}{版本 3.02 正式发布}

% \begin{change}
%   \item 修复名字系列命令自定义格式时出现的空格问题,比如 \lstinline{\listfigurename}。
%   \item 英文定理类名字改为中文名。
%   \item 英文结构名改为中文。
% \end{change}

% \datechange{2018/12/16}{版本 3.01 正式发布}

% \begin{change}
%   \item 调整 \lstinline{ctex} 宏包。
%   \item 说明文档增加更新内容。
% \end{change}

% \datechange{2018/12/06}{版本 3.00 正式发布}

% \begin{change}
%   \item 删除 \lstinline{mathpazo} 数学字体选项。
%   \item 添加邮箱命令 \lstinline{\mailto}。
%   \item 修改英文字体为 \lstinline{newtx} 系列,另外大型操作符号维持 cm 字体。
%   \item 中文字体改用 \lstinline{ctex} 宏包自动设置。
%   \item 删除 \lstinline{xeCJK} 字体设置,原因是不同系统字体不方便统一。
%   \item 定理换用 \lstinline{tcolobox} 宏包定义,并基本维持原有的定理样式,优化显示效果,支持跨页;定理类名字重命名,如 etheorem 改为 theorem 等等。
%   \item 删去自定义的缩进命令 \lstinline{\Eindent}。
%   \item 添加参考文献宏包 \lstinline{natbib}。
%   \item 颜色名字重命名。
% \end{change}

% \nocite{*}
% \printbibliography[heading=bibintoc, title=\ebibname]
% \appendix

% \chapter{基本数学工具}


% 本附录包括了计量经济学中用到的一些基本数学,我们扼要论述了求和算子的各种性质,研究了线性和某些非线性方程的性质,并复习了比例和百分数。我们还介绍了一些在应用计量经济学中常见的特殊函数,包括二次函数和自然对数,前 4 节只要求基本的代数技巧,第 5 节则对微分学进行了简要回顾;虽然要理解本书的大部分内容,微积分并非必需,但在一些章末附录和第 3 篇某些高深专题中,我们还是用到了微积分。

% \section{求和算子与描述统计量}

% \textbf{求和算子} 是用以表达多个数求和运算的一个缩略符号,它在统计学和计量经济学分析中扮演着重要作用。如果 $\{x_i: i=1, 2, \ldots, n\}$ 表示 $n$ 个数的一个序列,那么我们就把这 $n$ 个数的和写为:

% \begin{equation}
% \sum_{i=1}^n x_i \equiv x_1 + x_2 +\cdots + x_n
% \end{equation}

\printbibliography[heading=bibintoc, title=\ebibname]
\appendix
\chapter{细节说明}
\section{环境配置}
\subsubsection{EDK2}\label{appendix:launchuefi}
以下是一个编译EDK2的OVMF和相关efi的示例脚本:
\begin{lstlisting}[language=bash]
#!/bin/bash

# 获取脚本所在目录的绝对路径
SCRIPT_DIR="$(cd "$(dirname "${BASH_SOURCE[0]}")" && pwd)"
# 获取项目根目录(假设脚本在 tools/scripts 目录下)
PROJECT_ROOT="$(cd "${SCRIPT_DIR}/../.." && pwd)"

cd ${PROJECT_ROOT}/edk2

export WORKSPACE=${PROJECT_ROOT}/edk2
export EDK_TOOLS_PATH=${PROJECT_ROOT}/edk2/BaseTools
export CONF_PATH=${PROJECT_ROOT}/tools/config/edk2 # 默认是 edk2/Conf ,为了优雅这里把它单独提出来

source edksetup.sh

make -C BaseTools -j$(nproc)

build

build -p ShellPkg/ShellPkg.dsc

build -p YourPkg/YourPkg.dsc

\end{lstlisting}
以下是一个一键启动EDK2 UEFI环境的示例脚本:
\begin{lstlisting}[language=bash]
#!/bin/bash

# 获取脚本所在目录的绝对路径
SCRIPT_DIR="$(cd "$(dirname "${BASH_SOURCE[0]}")" && pwd)"
# 获取项目根目录(假设脚本在 tools/scripts 目录下)
PROJECT_ROOT="$(cd "${SCRIPT_DIR}/../.." && pwd)"

# 定义资源文件路径
OVMF_CODE="${PROJECT_ROOT}/edk2/Build/Ovmf3264/DEBUG_GCC5/FV/OVMF_CODE.fd"
OVMF_VARS="${PROJECT_ROOT}/edk2/Build/Ovmf3264/DEBUG_GCC5/FV/OVMF_VARS.fd"
# SHELL_EFI="${PROJECT_ROOT}/edk2/Build/Shell/DEBUG_GCC5/X64/ShellPkg/Application/Shell/EA4BB293-2D7F-4456-A681-1F22F42CD0BC/DEBUG/Shell.efi"
RAW_ACPIVIEW_EFI="${PROJECT_ROOT}/edk2/Build/Shell/DEBUG_GCC5/X64/ShellPkg/Application/AcpiViewApp/AcpiViewApp/DEBUG/AcpiViewApp.efi"
HELLO_WORLD_EFI="${PROJECT_ROOT}/edk2/Build/YourPkg/DEBUG_GCC5/X64/YourPkg/Application/HelloWorld/HelloWorld/DEBUG/HelloWorld.efi"
HALLO_WORD_EFI="${PROJECT_ROOT}/edk2/Build/YourPkg/DEBUG_GCC5/X64/YourPkg/Application/HalloWord/HalloWord/DEBUG/HalloWord.efi"
MY_ACPIVIEW_EFI="${PROJECT_ROOT}/edk2/Build/YourPkg/DEBUG_GCC5/X64/YourPkg/Application/AcpiView/AcpiView/DEBUG/AcpiView.efi"

# 检查上述文件是否存在
RESOURCE_LIST=("$OVMF_CODE" "$OVMF_VARS" "$RAW_ACPIVIEW_EFI" "$HELLO_WORLD_EFI" "$HALLO_WORD_EFI" "$MY_ACPIVIEW_EFI")

for RESOURCE in "${RESOURCE_LIST[@]}"; do
	echo "检查文件: $RESOURCE"
	if [ ! -f "$RESOURCE" ]; then
		echo "错误:$RESOURCE 不存在,请确认编译路径是否正确"
		exit 1
	fi
done
# 创建运行目录(如果不存在)
PLAYGROUND_DIR="${PROJECT_ROOT}/playground"
rm -rf "$PLAYGROUND_DIR"
mkdir -p "$PLAYGROUND_DIR"

# 复制OVMF变量文件(避免修改原始文件)
cp "$OVMF_VARS" "${PLAYGROUND_DIR}/OVMF_VARS.fd"

mkdir -p "$PLAYGROUND_DIR/uefi"
# cp "$SHELL_EFI" "${PLAYGROUND_DIR}/uefi/Origin_Shell.efi"
cp "$RAW_ACPIVIEW_EFI" "${PLAYGROUND_DIR}/uefi/O_AcpiViewApp.efi"
cp "$MY_ACPIVIEW_EFI" "${PLAYGROUND_DIR}/uefi/My_AcpiView.efi"
cp "$HELLO_WORLD_EFI" "${PLAYGROUND_DIR}/uefi/HelloWorld.efi"
cp "$HALLO_WORD_EFI" "${PLAYGROUND_DIR}/uefi/HalloWord.efi"

# 暂停
# read -p "按任意键继续..."

# 启动QEMU进入UEFI shell
qemu-system-x86_64 \
	-machine q35,accel=kvm \
	-m 8G \
	-smp 4 \
	-drive if=pflash,format=raw,unit=0,file="${OVMF_CODE}",readonly=on \
	-drive if=pflash,format=raw,unit=1,file="${PLAYGROUND_DIR}/OVMF_VARS.fd" \
	-drive file=fat:rw:"${PLAYGROUND_DIR}/uefi",format=raw,if=ide,index=0 \
	-nographic \
	-no-reboot \
	-serial mon:stdio

\end{lstlisting}

\end{document}
